%!TEX TS-program = xelatex
%!TEX encoding = UTF-8 Unicode

\documentclass[12pt]{extarticle}
% extarticle is like article but can handle 8pt, 9pt, 10pt, 11pt, 12pt, 14pt, 17pt, and 20pt text

\def \ititle {Origins of Mind}
 
\def \isubtitle {Lecture 07}
 
\def \iauthor {Stephen A. Butterfill}
\def \iemail{s.butterfill@warwick.ac.uk}
\date{}

%for strikethrough
\usepackage[normalem]{ulem}

\input{$HOME/Documents/submissions/preamble_steve_handout}

%\bibpunct{}{}{,}{s}{}{,}  %use superscript TICS style bib
%remove hanging indent for TICS style bib
%TODO doesnt work
\setlength{\bibhang}{0em}
%\setlength{\bibsep}{0.5em}


%itemize bullet should be dash
\renewcommand{\labelitemi}{$-$}

\begin{document}

\begin{multicols}{3}

\setlength\footnotesep{1em}


\bibliographystyle{newapa} %apalike

%\maketitle
%\tableofcontents




%--------------- 
%--- start paste
\def \ititle {Origins of Mind}
 
\def \isubtitle {Lecture 07}
 
 
 
\
 
 
 
\begin{center}
 
{\Large
 
\textbf{\ititle}: \isubtitle
 
}
 
 
 
\iemail %
 
\end{center}
 
 
 
How do humans first come to communicate with words?

 
 
 
\section{Preview: Shipwreck Survivor vs Lab Rat}
 
 
‘children learn words through the exercise of reason’ 
(\citealp[p.\ 1103]{Bloom:2001ka}; see \citealp{Bloom:2000qz})
 
‘Augustine describes the learning of human language as if the child came into a strange country and did not understand the language of the country; that is, as if it already had a language, only not this one.  Or again: as if the child could already think, only not yet speak.’
\citep[15--16, §32]{Wittgenstein:1953mm}
 
‘[t]he child learns this language from the grown-ups by being trained to its use. I am using the word ‘trained’ in a way strictly analogous to that in which we talk of an animal being trained to do certain things. It is done by means of example, reward, punishment, and suchlike’
\citep[p.\ 77]{Wittgenstein:1972lj}
 
‘the child’s early learning of a verbal response depends on society's reinforcement of the response in association with the stimulations that merit the response’
(\citep[p.\ 82]{Quine:1960fe}; compare \citep[pp.\ 28--9]{Quine:1974rd})
 
‘A child learning to speak is learning habits and associations which are just as much determined by the environment as the habit of expecting dogs to bark and cocks to crow’
\citep[p.\ 71]{Russell:1921ww}
 
Children acquiring language create their own words before they learn to use those of the adults around them.
 
‘Some children are so impatient that they coin their own demonstrative pronoun.  For instance, at the age of about 12 months, Max would point to different objects and say “doh?,” some¬times with the intent that we do something with the objects, such as bring them to him, and sometimes just wanting us to appreciate their existence’
(\citealp[p.\ 122]{Bloom:2000qz}; see further \citealp{Clark:1981bi,Clark:1982hj}).
 
Even where children have mastered a lexical convention, they will readily violate it in their own utterances in order to get a point across.
 
‘From the time they first use words until they are about two or two-and-a-half, children noticeably and systematically overextend words.  For example, one child used the word “apple” to refer to balls of soap, a rubber-ball, a ball-lamp, a tomato, cherries, peaches, strawberries, an orange, a pear, an onion, and round biscuits’
\citep[p.\ 35]{Clark:1993bv}
 
Children with no experience of others' languages can create their own languages.
\citep{Kegl:1999es,Senghas:2001zm,Goldin-Meadow:2003pj}
 
 
 
\section{Does being able to think depend on being able to communicate with language?}
 
\begin{enumerate}
 
\item
 
If someone can think, she must be capable of having a false belief.
 
\item
 
To be capable of having a false belief it is necessary to understand the possibility of false belief.
 
\item
 
Understanding the possibility of false belief entails being able to communicate by language.
 
\end{enumerate}
 
'belief is central to all kinds of thought.  If someone is glad that, or notices that, or remembers that, or knows that, the gun is loaded, then he must believe that the gun is loaded.  Even to wonder whether the gun is loaded, or to speculate on the possibility that the gun is loaded, requires belief, for example, that a gun is a weapon, that it is a more or less enduring physical object, and so on.  …  it is necessary that there be endless interlocked beliefs'
\citep[p.\ 157]{Davidson:1975eq}; cf. \citep[pp.\ 320--1]{Davidson:1982je}
 
‘We, observing and describing … a creature …, say that it discriminates certain shapes, objects, colors, and so forth, by which we mean that it reacts in ways we find similar to shapes, objects, and colors which we find similar.

But we would be making a mistake if we were to assume that because the creature discriminates and reacts in much the way we do, that it has the corresponding concepts.The difference, as I keep emphasizing, lies in the fact that we, unlike the creature I am describing, can, from our point of view, make mistakes in classification.’
Davidson (2001: 11)

\citep[p.\ 11]{Davidson:2001np}
 
‘we grasp the concept of truth only when we can communicate the contents---the propositional contents---of the shared experience, and this requires language’
\citep[p.\ 27]{Davidson:1997wj}.
 
‘the process of language acquisition [is] coming to know the meanings of words, where at a given stage the learner’s conception is an hypothesis about the meaning’
\citep[p.\ 153]{Higginbotham:1998rm}
 
 
 
\section{Training}
 
‘The ability to discriminate, to act differentially in the face of clues to the presence of food, danger or safety, is present in all animals and does not require reason.  Nor does the learning, even of complex routines, require reason, for it is possible to learn how to act without learning that anything is the case.’
\citep[p.\ 326]{Davidson:1982je}
 
‘A child learning to speak is learning habits and associations which are just as much determined by the environment as the habit of expecting dogs to bark and cocks to crow’
\citep[p.\ 71]{Russell:1921ww}
 
‘[t]he child learns this language from the grown-ups by being trained to its use. I am using the word ‘trained’ in a way strictly analogous to that in which we talk of an animal being trained to do certain things. It is done by means of example, reward, punishment, and suchlike’
\citep[p.\ 77]{Wittgenstein:1972lj}
 
‘the child’s early learning of a verbal response depends on society's reinforcement of the response in association with the stimulations that merit the response’
(\citep[p.\ 82]{Quine:1960fe}; compare \citep[pp.\ 28--9]{Quine:1974rd})
 
‘Before we have an idea of truth or error, before the advent of concepts or propositional thought,
there is a rudiment of communication in the simple discovery that sounds produce results. Crying is the first step toward language when crying is found to procure one or another form of relief or satisfaction. More specific sounds, imitated or not, are rapidly associated with more specific pleasures.
Here use //p. 71// would be meaning, if anything like intention and meaning were in the picture.
A large further step has been taken when the child notices that others also make distinctive sounds at the same time the child is having the experiences that provoke its own volunteered sounds.
For the adult, these sounds have a meaning, perhaps as one word sentences. The adult sees herself as doing a little ostensive teaching: “Eat,” “Red,” “Ball,” “Mamma,” “Milk,” “No.”
 
There is now room for what the adult views as error: the child says “Block” when it is a slab. This move fails to be rewarded, and the conditioning becomes more complex’
\citep[pp.\ 70--1]{Davidson:2000mt}
 
 
 
\section{Understanding}
 
‘You can deceive yourself into thinking that the child is talking if it makes sounds which, if made by a genuine language-user, would have a definite meaning.
…  If a mouse had vocal cords of the right sort, you could train it to say “Cheese”. 
But that word would not have a meaning when uttered by the mouse,
nor would the mouse understand what it “said”
.’
\citep[p.\ 11]{Davidson:1999ju}
 
'to attribute to a speaker no more than knowledge of how to play … interlocking language games is to make him a participant in an activity he cannot survey (‘cannot see what is going on’).'
\citep[p.\ 224]{Dummett:1979fb}
 
Understanding a word can’t be purely a practical ability because this would ‘render mysterious our capacity to know whether we are understanding.’
\citep[p.\ 93]{Dummett:1991yj}
 
Language is ‘a rational activity on the part of creatures to whom can be ascribed intention and purpose’.
So we need to distinguish ‘those regularities of which a language speaker, acting as a rational agent engaged in conscious, voluntary action, makes use from those that may be hidden from him.’
\citep[p.\ 104]{Dummett:1978zv}
 
‘A child at this stage has no linguistic knowledge but merely a training in certain linguistic practices.  When he has reached a stage at which it is possible for him to lie, his utterances will have ceased to be merely responses to features of his environment or to experienced needs.  They will have become purposive actions based upon a knowledge of their significance to others.’
\citep[p.\ 95]{Dummett:1991ug}
 
 
 
\section{Creativity}
 
‘Intentional action cannot emerge before belief and desire, for an intentional action is one explained by beliefs and desires that caused it.’
\citep[p.\ 10]{Davidson:1999ju}
 
 
 
\section{Mapping words to concepts}
 
‘children learn words through the exercise of reason’ 
(\citealp[p.\ 1103]{Bloom:2001ka}; see \citealp{Bloom:2000qz})
 
‘much of what goes on in word learning is establishing a correspondence between the symbols of a natural language and concepts that exist prior to, and independently of, the acquisition of that language’
 
\citep[p.\ 242]{Bloom:2000qz}
 
‘to know the meaning of a word is to have: 

1. a certain mental representation or concept 

2.	that is associated with a certain form’

\citep[p.\ 17]{Bloom:2000qz}
 
‘Augustine describes the learning of human language as if the child came into a strange country and did not understand the language of the country; that is, as if it already had a language, only not this one.  Or again: as if the child could already think, only not yet speak.’
\citep[15--16, §32]{Wittgenstein:1953mm}
 
‘Augustine’s proposal is no longer seen as the goofy idea that it once was’ \citep[p.\ 61]{Bloom:2000qz}.
 
'The tutor names things in accordance with the semantic customs of the community.  The player forms hypotheses about the categorical nature of the things named.  He tests his hypotheses by trying to name new things correctly.  The tutor compares the player's utterances with his own anticipations of such utterances and, in this way, checks the accuracy of fit between his own categories and those of the player.  He improves the fit by correction.'
(Brown 1958, p. 194 as quoted by \citep[p.\ 19]{Clark:1993bv})
 
‘One of the first problems children take on is the MAPPING of meanings onto forms …  They must identify possible meanings, isolate possible forms, and then map the meanings onto the relevant forms.’
\citep[p.\ 14]{Clark:1993bv}
 
 
 
 
 
 
\section{Appendix: Grice/Tomasello (optional)}
 
‘children acquire linguistic symbols as a kind of by-product of social interaction with adults’
\citep[p.\ 90]{Tomasello:2003fk}
 
Infants ‘begin to comprehend and use … linguistic symbols on the basis of their skills of social cognition and cultural learning’
\citep[p.\ 582]{Tomasello:1999en}
 
‘language is itself one type-albeit a very special type-of joint attentional skill’
\citep[p.\ 1120]{Tomasello:2001ic}
 
‘the kind of rational activity which the use of language involves is a form of rational cooperation’
\citep[p.\ 341]{Grice:1989ha}
 
‘it is an error to suppose we have seen deeply into the heart of linguistic communication when we have noticed how society bends linguistic habits to a public norm.
…  But in indicating this element of the conventional, or of the conditioning process that makes speakers rough linguistic facsimiles of their friends and parents, we explain no more than the convergence; we throw no light on the essential nature of the skills that are thus made to converge.’
\citep[p.\ 278]{Davidson:1982uu}
 
‘convention does not help explain what is basic to linguistic communication, though it may describe a usual, though contingent feature.’
\citep[p.\ 280]{Davidson:1982uu}
 
‘An utterance has certain truth conditions only if the speaker intends it to be interpreted as having those truth conditions.
Moral, social or legal considerations may sometimes invite us to deny this, but I do not think the reasons for such exceptions reveal anything of importance about what is basic to communication’
\citep[p.\ 310]{Davidson:1990du}
  
%--- end paste
%--------------- 
 
\footnotesize 
\bibliography{$HOME/endnote/phd_biblio}

\end{multicols}

\end{document}