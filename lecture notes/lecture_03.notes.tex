%!TEX TS-program = xelatex
%!TEX encoding = UTF-8 Unicode

%\def \papersize {a5paper}
\def \papersize {a4paper}
%\def \papersize {letterpaper}

%\documentclass[14pt,\papersize]{extarticle}
\documentclass[12pt,\papersize]{extarticle}
% extarticle is like article but can handle 8pt, 9pt, 10pt, 11pt, 12pt, 14pt, 17pt, and 20pt text

\def \ititle {Origins of Mind: Lecture Notes}
\def \isubtitle {Lecture 01}
%comment some of the following out depending on whether anonymous
\def \iauthor {Stephen A.\ Butterfill}
\def \iemail{s.butterfill@warwick.ac.uk% \& corrado.sinigaglia@unimi.it
}
%\def \iauthor {}
%\def \iemail{}
%\date{}

%\input{$HOME/Documents/submissions/preamble_steve_paper4}
\input{$HOME/Documents/submissions/preamble_steve_lecture_notes}

%no indent, space between paragraphs
\usepackage{parskip}

%comment these out if not anonymous:
%\author{}
%\date{}

%for e reader version: small margins
% (remove all for paper!)
%\geometry{headsep=2em} %keep running header away from text
%\geometry{footskip=1.5cm} %keep page numbers away from text
%\geometry{top=1cm} %increase to 3.5 if use header
%\geometry{bottom=2cm} %increase to 3.5 if use header
%\geometry{left=1cm} %increase to 3.5 if use header
%\geometry{right=1cm} %increase to 3.5 if use header

% disables chapter, section and subsection numbering
\setcounter{secnumdepth}{-1} 

%avoid overhang
\tolerance=5000

%\setromanfont[Mapping=tex-text]{Sabon LT Std} 


%for putting citations into main text (for reading):
% use bibentry command
% nb this doesn’t work with mynewapa style; use apalike for \bibliographystyle
% nb2: use \nobibliography to introduce the readings 
\usepackage{bibentry}

%screws up word count for some reason:
%\bibliographystyle{$HOME/Documents/submissions/mynewapa} 
\bibliographystyle{apalike} 


\begin{document}



\setlength\footnotesep{1em}






%--------------- 
%--- start paste

      
\title {Origins of Mind: Lecture Notes \\ Lecture 03}
 
 
 
\maketitle
 
\subsection{slide-3}
The main theme of this lecture is knowledge of causal interactions.
My question is: How do humans first acquire knowledge of particular causal 
interactions among objects.
I want to approach this question in an odd way, by thinking about the perception of causation
in adults.
But before I come to that, I also want to mention that there is a second theme for this lecture.
 
\subsection{slide-5}
In this lecture I want to consider several interlocking themes simultaneously.
The second theme is the Problem I introduced in the last lecture, which arises from
the failure of the Simple View.
The Problem is to understand the relation between infants' abilities regarding physical objects
and the Spelke principles which describe these given that the principles and their deliverances
are not actually knowledge.
It turns out that thinking about the perception of causation will help us with this theme too.
 
Now since the last lecture was a week ago, 
let me start by giving you a very recap.
What is the Simple View, why does it fail and why is its failure a problem?
 
\subsection{unit\_221}
 
 
\section{Recap: The Simple View, Not}
 
\subsection{slide-7}
Knowledge of objects depends on abilities to (i) segment objects, (ii) represent them as 
persisting and (iii) track their interactions.
 
\emph{Question 1}  How do humans come to meet the three requirements on knowledge of objects?
 
Until quite recently it was held, following Piaget and others, that these three abilities 
appeared relatively late in development.  
However, as we saw last week, more recent investigations provide strong evidence that all three 
abilities are present in humans from around four months of age or earlier.  
Infants' looking behaviours indicate that they have expectations concerning segmentation, 
persistence and causal interactions.
 
\emph{Discovery 1} Infants manfiest all three abilities from around four months of age or 
earlier.
 
\subsection{slide-8}
We also saw that infants' abilities to segement objects, represent them as persisting and track 
their causal interactions can be described by appeal to a single set of principles, 
the principles of cohension, boundedness, rigidity and no action at a distance.
 
This suggests that 
          
\emph{Discovery 2} Although abilities to segment objects, to represent them as persisting 
through occlusion and  to track their causal interactions are conceptually distinct, they 
may all be consequences of a single mechanism (in humans and perhaps in other animals).
          
 
Spelke suggests, further, that these principles of object perception explain infants' looking 
behaviours.
 
This means we must ask
          
\emph{Question 2} What is the relation between the principles of object perception and infants’ looking behaviours?
 
\subsection{slide-9}
In answer to Q2, I suggested that we start with the Simple View.
 
The \emph{Simple View} is the view that the principles of object perception are things that we know or believe, and we generate expectations from these principles by a process of inference.
 
The attraction of the simple view is that it promises to explain infants' sensitivity to 
objects' boundaries, their persistence and their causal interactions as manifested in a variety 
of looking behaviours.
 
\subsection{slide-10}
The problem for the simple view is that it makes exactly the wrong prediction about infants' 
reaching behaviours.
 
This problem arises twice, once for object permanence and again for causal interactions.
Take object permanence first.  In this case the problem is 
that four and five month old infants appear to be surprised when a solid object is placed 
behind a drawbridge 
and then the drawbridge rotates back 180 degrees as if the object wasn't there.  
But if you put a desirable object behind a screen, these infants won't ever search behind the 
screen for the object. 
So if we measure their looking behaviours, they seem to know that an object is behind the 
screen; but if we measure their searching behaviours they seem not to know this.
 
\subsection{slide-11}
As I keep saying, 
the problem for the Simple View is that it correctly predicts looking behaviours but not 
searching behaviours.
This problem arises a second time in the case of infants' abilities to track causal 
interactions.
Suppose you roll a ball down this ramp and remove it from behind one of the doors.
You can predict where the ball will stop because you can see this barrier sticking up here.
2 and 2.5 year olds will look longer if you remove it from any door other than the right one.
These children and infants can also predict the location of the object (not just identify a 
violation, but look forward to where the object is) \citep{mash:2006_what}.
But if you ask them to remove the ball (and offer them a reward for doing so), 
the two year olds will typically choose a favourite door to open every time regardless of where 
the barrier is.  
(Incidentally this is not random but seems to be based on momentum 
\citep{perry:2008_representational}.) 
And the 2.5 year olds will choose a door adjacent to the barrier but do not prefer the door on 
the correct side of the barrier.
 
The failure to reach is a problem for the Simple View because it predicts that children will 
correctly retrieve the ball.
 
\subsection{slide-12}
\emph{Discovery 3}  The Simple View generates systematically false predictions (about reaching).
So we have to reject it.
 
We were led to the simple view by Question 2, What is the relation between the principles of 
object perception and infants' looking behaviours?
The failure of the Simple View leaves us with two questions ...
 
\emph{Question 2a} Given that the simple view is wrong, what is the relation between the principles of object perception and infants’ competence in segmenting objects, object permanence and tracking causal interactions?
 
\emph{Question 2b} The principles of object perception result in ‘expectations’ in infants.  What is the nature of these expectations?
 
\subsection{slide-13}
As a first move, you might say that infants do not know or believe the principles,
but they do represent them.  This move involves making some assumptions, but I think it is a 
good move as far as it goes.
The problem is that it doesn't go very far.
Representation is a highly abstract notion. 
Knowledge and belief are types of representation, and there are many other kinds.
In order to make predictions, we need to say what kind of representation underpins
infants' competence.
 
\subsection{slide-14}
Just here we have come up against what I'm calling Davidson's challenge.
The problem is that we have to describe something which is not mindless (because it does 
involve representation) but which is also not full-blown thought.
 
--------
\subsection{slide-17}
Will we fail?
I think not
 
\subsection{slide-18}
I am emphasising this problem so much because it is quite general.
It doesn't arise only in the case of knowledge of objects but also in other domains (like knowledge of number and knowledge of mind).
And it doesn't arise only from evidence about infants or nonhuman primates; it would also arise if our focus were exclusively on human adults.
More on this later.
For now, our aim is to search for solutions to the problem.
 
\subsection{slide-19}
Why is this important?
The discovery of infant abilities presents us with a dilemma.
If we follow researchers like Michael Tomasello in dismissing infants' abilities to focus on 
social interactions, culture and language, we miss something important about the origins of 
knowledge.  We will miss the fact that coming to know things about physical objects (and 
colours, and mind and more besides) involves multiple layers of representation.
But if we follow researchers like Spelke and Baillargeon is discussing infants' abilities as 
if they were based on knowledge, belief and inference, we will also miss something important 
about the origins of knowledge.  We will miss the fact that infants' abilities, although 
essential for later developments, are only indirectly related to the knowledge of physical 
objects that adult humans enjoy.
The challenge is to avoid this dilemma, to provide a theoretical framework that will allow us 
to understand how infants' abilities support the emergence of knowledge without themselves 
amounting to knowledge.
 
\subsection{slide-20}
I started this lecture by saying I want to examine two interlocking themes.  
We can make progress with the Problem by investigating how humans first acquire 
knowledge of causal interactions among particular objects.
 
\subsection{slide-22}
Actually there is a third theme that I want to tackle simultaneously with the other two.
 
\emph{Question 3} What is the relation between adults’ and infants’ abilities concerning physical objects and their causal interactions?
 
Let me explain ...
 
\subsection{slide-23}
If we accepted the Simple View, we would think of the relation between adult and infant
abilities like this (adult is just larger infant) ...
 
\subsection{slide-27}
Rejecting the Simple View does not force us to reject that picture of the relation between 
infant and adult cognitive abilities, but it does open up the possibility of an alternative 
view (the `two systems' view)
 
\subsection{slide-28}
So, at the risk of confusing you, this lecture is going to be about three things.
On the first, there are two issues:
 
Given that the simple view is wrong, what is the relation between the principles of object perception and infants’ competence in segmenting objects, object permanence and tracking causal interactions?
 
The principles of object perception result in ‘expectations’ in infants.  What is the nature of these expectations?
 
The second is just the question, How do humans first acquire knowledge of particular causal 
interactions among objects?
 
And the third is What is the relation between adults’ and infants’ abilities concerning physical objects and their causal interactions?
 
I'm going to approach these indirectly, by asking a question about perception of causation.
This won't seem relevant yet, but it will become relevant.
 
\subsection{unit\_251}
 
 
\section{Perception of Causation}
 
\subsection{slide-30}
The question for this section is,
 
Can humans perceive causal interactions?
 
\subsection{slide-31}
Let my try and show you stimuli that were used in an experiment (without yet telling you 
anything about the experiment).  What do you see?
 
[If the animation doesn't work, there's a static version on the next slide.]
 
\subsection{slide-32}
OK, so adults: (a) verbal reports.  So what?
 
‘There are some cases … in which a causal impression arises, clear, genuine, and unmistakable, 
and the idea of cause can be derived from it by simple abstraction in just the same way as the 
idea of shape or movement can be derived from the perception of shape or movement’ 
\citep[p.\ 270--1]{Michotte:1946nz}
 
\subsection{slide-33}
Adults will also report experiencing causal interactions including pullling, ...
 
\subsection{slide-35}
... disintegration ...
 
\subsection{slide-37}
... and bursting.
 
\subsection{slide-38}
This effect, or one very like
it, can also be found in infancy.
 
Of course, six-month-olds can't tell us about their expeirences.
So how can we tell that they detect launching effects?
 
\subsection{slide-39}
Infants at around six months of age seem also to distinguish launching from other sequences, 
much as adults do \citep{Leslie:1987nr}.
 
[nb: Several people have discussed this in seminars so I won't discuss it here (the reference is on your handout).]
 
In their experiment, they compared two groups of infants.
The first group was habituated to the top display, which is just the sort of animation Michotte
used to get reports of causal experiences in adults.
After habituation, this first group was then shown the same display except that the direction was 
reversed.
Meanwhile a second group was habituated to a display like the top display here except 
that there was a delay between the first object stopping and the second object starting.
This delay would mean that, in adults, there are no reports of experiences of a causal interaction.
After habituation, this second group was shown the same display except that the direction of 
movement was reversed.
 
Of interest was whether the first group showed greater dishabituation to the reversal than 
the second group.
How could this tell us anything about infants' experiences?
Suppose that infants do not have anything like what adults report as an experience of causation.
They they experience merely patterns of movement.
And, in this case, reversing on sequence should create no more interest than reversing the other.
But now suppose that infants do have something like what adults report as an experience 
of causation?
Then, when reversing the first sequence, there are two changes: there is a change both to the 
movement and to the character of the causal interaction.  To put it informally,
reversing direction means that the patient of the interaction becomes its agent.
So the hypothesis that infants' experiences of Michotte-like stimuli resemble adults 
predicts that there will be greater dishabituation when the first, `direct launching` sequence 
is reversed.
 
\subsection{slide-40}
And this is just what the researchers found.
The table shows mean looking times in second (with standard deviations in brackets).
The control group was just like the direct lanunching group except that there was no reversal
 
\subsection{slide-41}
So can humans, adult and infant, perceive causal interactions?
 
\subsection{slide-43}
So far I don't think we have strong reasons to accept that they do.
In infants we have discrimination and in adults we have verbal reports.
But we shouldn't trust verbal reports.
After all, people will say all kinds of things about their experiences.
This is nicely illustrated by a famous experiment on apparent behaviour by \citet{Heider:1944ts}.
 
\subsection{slide-45}
Michotte: the experience of launching depends on interactions among various factors including 


            

              
the relative speeds of the two objects

              
the delay between the first and second objects’ movements

              
the spatial gap between the two objects 

              
the trajectories of the two objects. 

            
 
But how does this help us?  Importantly, tiny variations in the parameters will make 
big differences in the experiences reported.
Let me illustrate this for the delay between the objects' movements.
 
\subsection{slide-46}
adults: (b) they can discriminate between short gaps and long gaps.
 
That is, the can discriminate gaps of around 50ms.
 
\subsection{slide-47}
Maybe this is clearer as a figure.
 
People can distinguish between stimuli that differ only in that the gap between two movements
is approximately 50ms longer in one than the other.
A 50ms difference makes the difference between reporting launching and reporting two movements.
 
We need to do more to understand the effect, ...
 
\subsection{slide-48}
The question was how we can get beyond intuition in understanding the verbal reports.
 
Part of the answer is this.  We don't worry about the content of the verbal reports.
We just focus on the fact that their content changes depending on a tiny, 50 millisecond
difference in the delay between two movements.
Call this \emph{launching effect}.
 
This doesn't tell us what people are detecting.
But it does tell us that the effect is not merely confabulation or making it up.
So we have taken a tiny step beyond intuition.  But we also have to answer two questions.
 
\subsection{slide-49}
How is launching detected?  For example, does it involve perceptual processes?

            
Why is a delay of up to around 70ms consistent with the launching effect occuring?
 
 
\subsection{slide-51}
We'll focus on the first question and come back to the second one later.
 
\subsection{slide-52}
So we have the launching effect: adults and probably infants too exhibit perceptual sensitivity 
to differences in timing of around 50 milliseconds, but only when such delays make the difference
between a causal interaction or non-causal interaction.
 
We are still trying to understand the nature of the launching effect.
To make progress we need to think about how it arises.
 
Guess  how the launching effect works!  
A natural thought is this: first you perceive objects, then you identify causal interactions 
based on contiguity etc.
This turns out to be completely wrong.
 
\subsection{slide-53}
The impression of launching is judgement-independent.
So it can't be a consequence of thinking about the interaction.
Still, it might be a consequence of perceiving objects in certain relations to each other.
However a key finding shows that this is wrong.
Surprisingly, we don't first perceive objects and then get the launching effect;
rather, the launching effect is tied up with perceptual process of identifying objects' surfaces.
 
\subsection{slide-54}
[I'm about to talk about illusory causal crescents.  I first show them two videos.
This is a full overlap video.
You can drag the slider to show them that it's full overlap, but first ask them what they see.]
[Static images follow in case video doesn't work.]
 
\subsection{slide-56}
Normally, if the two balls overlap completely, subjects report seeing a single object 
changing colour.
 
\subsection{slide-57}
[This is a causal capture video with full overlap.
Focus on the top sequence.  Tell me what you saw!
You can drag the slider to show them that it's full overlap, but first ask them what they see.]
[Static images follow in case video doesn't work.]
 
\subsection{slide-59}
Normally in this case people report the impression that the top sequence collided.
That is, they didn't pass, they collided.
 
\subsection{slide-60}
Causal capture is described by \citep{Scholl:2002eb}.
As I said,
normally, if the two balls overlap completely, subjects report seeing a single object 
changing colour.
But if we show subjects a sequence like the launching effect but where the first square 
overlaps the second's position before it moves.  When this event is shown is isolation 
almost all subjects see it as a single object changing colour.  But when the event is 
shown with an 
unambiguous launching effect nearby, almost all subjects now see the 'overlap' event as 
a launching.
Causal capture means that we can show subjects a sequence with complete overlap and 
still have the report a causal effect.
 
Why do we care about causal capture?  Because it gives rise to illusory causal 
crescents ...
 
\subsection{slide-61}
[Now I'll explain illusory causal cresecents.]
 
Here's a static image representing the sequence you saw first, when there was full overlap.
 
\subsection{slide-63}
‘when there is a launching event beneath the overlap (or underlap event) timed such that 
the launch occurs at the point of maximum overlap, observers inaccurately report that 
the overlap is incomplete, suggesting that they see an illusory crescent.’ 
\citep[p.\ 461]{Scholl:2004dx}
 
Why does the illusory causal crescent appear?  Scholl and Nakayama suggest a  
‘a simple categorical explanation for the Causal Crescents illusion: the visual system, 
when led by other means to perceive an event as a causal collision, effectively 
‘refuses’ to see the two objects as fully overlapped, because of an internalized 
constraint to the effect that such a spatial arrangement is not physically possible. 
As a result, a thin crescent of one object remains uncovered by the other one-as 
would in fact be the case in a straight-on billiard-ball collision where the motion 
occurs at an angle close to the line of sight.’ 
\citep[p.\ 466]{Scholl:2004dx}
 
*here or later?
Contrast Spelke’s view.
‘objects are conceived: Humans come to know about an object’s unity, boundaries, and 
persistence in ways like those by which we come to know about its material composition 
or its market value.’
\citep[p.\ 198]{Spelke:1988xc}.
 
\subsection{slide-64}
(*This figure just shows when the overlap event was perceived as causal.)
 
\subsection{slide-65}
The question was,
How is launching detected?  For example, does it involve perceptual processes?
 
We've just taken a step towards answering this first question.  As 
\citet[p.\ 299]{Scholl:2000eq} put it,
 
‘just as the visual system works to recover the physical structure of the world by inferring 
properties such as 3-D shape, so too does it work to recover the causal …  structure of 
the world by inferring properties such as causality’ 
\citep[p.\ 299]{Scholl:2000eq}
 
But more is needed ...
 
\subsection{slide-68}
The second question, Why is a delay of up to around 70ms consistent with the 
launching effect occuring?, remains open.
 
\subsection{slide-69}
Summary so far
 
The question for this section was:
Can humans perceive causal interactions?
 
Adults report experiencing a launching effect.

          
These reports show that they can make minute distinctions.
They can detect tiny, 50 millisecond differences in the delay between two movements.

          
Infants detect apparent causal interactions from 6 months of age or earlier.

          
Detecting apparent causal interactions is bound up with perceiving objects.
It's not that we first perceive objects and surfaces, and then identify causal interactions.
Rather, identifying causal interactions is part of perceiving surfaces and objects.  
So the launching effect does seem to involve perceptual processes.
 
They can detect tiny, 50 millisecond differences in the delay between two movements.
 
It's not that we first perceive objects and surfaces, and then identify causal interactions.
Rather, identifying causal interactions is part of perceiving surfaces and objects.  
So the launching effect does seem to involve perceptual processes.
 
This fourth point suggests we should step back and consider how objects are perceived.
That is what we'll do next.
 
\subsection{unit\_261}
 
 
\section{Object Indexes and Causal Interactions}
 
\subsection{slide-72}
(from figure caption): ' A number (here eight) of identical objects are shown (at t = 1), 
and a subset (the `targets') is selected by, say,  ̄ashing them (at t 􏰈 2), after which 
the objects move in unpredictable ways (with or without self-occlusion) for about 10 s. 
At the end of the trial the observer has to either pick out all the targets using a 
pointing device or judge whether one that is selected by the experimenter (e.g. by 
 ̄ashing it, as shown at t 􏰈 4) is a target.' \citep[p.\ 142]{Pylyshyn:2001hl}
 
Highlight the case where subject is asked whether this is one of the objects identified.
(If a target disappears, subjects can also say where it was and which direction it was 
moving in.)
 
The limit of 3, maybe 4, objects will be important later.
 
\subsection{slide-78}
What does this tell us?
 
If attention is organised around objects, the perceptual system must be capable of identifying and tracking objects.
 
\subsection{slide-79}
Leslie et al say an object index is ‘a mental token that functions as a pointer to an 
object’ \citep[p.\ 11]{Leslie:1998zk}
 
‘Pylyshyn’s FINST model: you have four or five indexes which can be attached to objects; 
it’s a bit like having your fingers on an object: you might not know anything about the object, but you can say where it is relative to the other objects you’re fingering. (ms. 19-20)’ \citep{Scholl:1999mi}
 
\subsection{slide-84}
Wait a minute, does this remind you of anything?  Hopefully!
 
\subsection{slide-87}
But before we get to that, what is this about causal interactions?
It's reasonably obvious that object indexes require segmentation (otherwise how would they
attach to *objects*), and the evidence that they can survive occlusion is relatively easy to
understand.  But the point that having object indexes involves tracking causal interactions
is less straightforward.  So let's consider that ...
 
\subsection{slide-88}
Object indexes are linked to causation.
In order to track objects, a perceptual system has to be sensitive to be causal interactions
 
Why is this true?
Because when you have a causal interaction, there's a conflict between principles of object 
perception e.g. distinct surfaces=>two objects, vs good continuity of motion=>one object
The perceptual system needs to know when conflicts should be reconciled and when they should 
be written off.
We get perceptual effects of causal interactions when there are conflicts among cues of object 
identity.
 
This is a point Michotte made.  He found that launching occurs when there is a conflict between 
cues to object identity: good continuity of movement suggests a single object whereas the 
existence of two distinct surfaces indicates two objects.
 
It is plausible that other types of causal interaction also involve conflicts between cues 
to object identity.
 
\subsection{slide-89}
Recall the question I asked earlier about 70ms.
 
Why is a delay of up to around 70ms consistent with the launching effect occuring?
 
This is an important question insofar as we are concerned with detecting causal interactions.
Is what people detect when the launching effect occurs a causal interaction?
You might say, it can't be a causal interaction 
because no delay between two movements is consistent with a causal interaction.
 
\subsection{slide-90}
Michotte said this:
 
‘anyone not very familiar with the procedure involved in framing the physical concepts of 
inertia, energy, conservation of energy, etc., might think that these concepts are simply 
derived from the data of immediate experience.’
\citep[p.\ 223]{Michotte:1946nz}
 
How is this consistent with the laws of mechanics—surely no pause can be tolerated? 
Ingeniously, Michotte compares launching with the movement of a single object. The single 
object moves half way across a screen then pauses before continuing to move. Michotte found 
that the longest pause between the two movements consistent with subjects experiencing them 
as a single movement is around 80ms, exactly the longest pause consistent with experiences 
characteristic of launching \citep[pp.\ 91--8, 124]{Michotte:1946nz}. Accordingly, the 
experience characteristic of launching appears to require that the two movements be 
experienced as uninterrupted—--this is why they can be separated by a pause of up to but 
no longer than 80ms.
 
\subsection{slide-91}
The question for this section was:
Can humans perceive causal interactions?
 
Now I think we have achieved an answer:
 
Perceptual systems identify certain kinds of causal interaction in the course of 
tracking objects.
 
The perceptual system responsible for identifying objects must also concern itself with certain 
kinds of causal interaction in order to reconcile conflicting cues to object identity.
 
In slightly more detail: one function of our perceptual systems is to identify and track 
objects; this is done by means of various cues; sometimes the visual system is faced with 
conflicting cues to object identity which need to be resolved in order to arrive at a 
satisfactory interpretation; when certain types of causal interaction occur there is a 
conflict among cues to object identity; these conflicts must be treated differently from other 
conflicts because they do not indicate failures of object identification and so do not  
require resolution or further perceptual processing. 
So object perception depends on sensitivity to certain types of causal interaction and this is 
why the launching effect occurs.
 
\subsection{slide-92}
Before concluding I want to mention some further evidence for this view.
 
\subsection{slide-93}
This further evidence exploits something called the object-specific preview effect.
 
So before I can go on, I need to explain what this is.
 
\subsection{slide-94}
Background: object-specific preview effect
 
We can measure object indexes using the object-specific preview effect.
 
The \emph{object-specific preview effect}: ‘observers can identify target letters that matched the preview letter from the same object faster than they can identify target letters that matched the preview letter from the other object.’
\citep[p.\ 2]{Krushke:1996ge}
 
\subsection{slide-99}
Krushke and Fragassi (1996) have shown that the object-specific preview effect vanishes 
in launching but not in various spatio-temporally similar sequences.  Since the object-specific 
preview effect is regarded as diagnostic of feature binding, this is evidence that in launching 
sequences, features of the second object (such as motion) remain bound to the first object for 
a short time after the second object starts to move.
 
\subsection{slide-100}
This is unexpected insofar as perception is often supposed to be limited to features of the 
world less abstract that causal interactions.  Indeed, the notion that perceptual processes
represent three-dimensional objects rather than mere surfaces was at one time controversial.
The research we have reviewed shows that perceptual processes represent not only 
three-dimensional but properly physical objects, that is, objects capable of causally
interacting with each other.
 
As we'll see in a moment, this is relevant to understanding the Problem we encountered 
in the lecture on objects.
 
\subsection{unit\_266}
 
 
\section{Object Indexes and the Principles of Object Perception}
 
\subsection{slide-102}
This is a lecture about the origins of our knowledge of causal interactions, but I want to 
return to the topic of objects and the problem we encountered in the last lecture.
 
As I keep saying, knowledge of objects depends on abilities to (i) segment objects, 
(ii) represent them as persisting and (iii) track their interactions.
 
\subsection{slide-103}
When we asked how infants meet these three requirements, we found that a single set of principles,
the Principles of Object Perception, seemed to underlie all three abilities.
 
We were then led to the question, What is the status of these principles?
It's one thing to say that they describe how infants perform; but what we want is some
understanding of the mechanisms.
 
\subsection{slide-104}
The Simple View is one way to get a mechanism out of the principles.
Recall that the \emph{simple view} is the view that the principles of object perception are things that we know or believe, and we generate expectations from these principles by a process of inference..
 
\subsection{slide-106}
Unfortunately, as we saw, the Simple View is wrong.
We know it is wrong because it makes systematically incorrect predictions about infants' actions.
These arise from a discrepancy between measures that involve looking times or eye movements
and measures that involve 
other kinds of action, such as searching and pulling.
 
At the end of the last lecture (on objects), I said that the failure of the Simple View 
leaves us with a problem.
We are now, at least, in a position to take a step towards solving that problem.
 
\subsection{slide-107}
The principles of object perception


            
are not items of knowledge 


            
instead 


            
they characterise the operation of 


            
object-indexes (aka FINSTs, mid-level object files)

Their upshot is not knowledge about particular objects and their movements but rather a 
perceptual representation involving an object index.
 
:t()
            reflecting on \citet{mccurry:2009_beyond} in one of the seminars ... distinguished
            initiating action and continuing to perform an action ... object indexes support
            guidance of action but not its initiation.
\citep{Leslie:1998zk,Scholl:1999mi,Carey:2001ue}.
 
This amazing discovery is going to take us a while to fully digest.  As a first step, note its
significance for Davidson's challenge about characterising what is going on in the head of the 
child who has a few words, or even no words.
 
\subsection{slide-108}
We saw this quote in the first lecture ...
 
The discovery that the principles of object perception characterise the operation of 
object-indexes doesn't mean we have met the challenge exactly.
We haven't found a way of describing the processes and representations that underpin infants'
abilities to deal with objects and causes.
However, we have reduced the problem of doing this to the problem of characterising how 
some perceptual mechanisms work.  
And this shows, importantly, that understanding infants' minds is not something different from
understanding adults' minds, contrary to what Davidson assumes.
The problem is not that their cognition is half-formed or in an intermediate state.
The problem is just that understanding perception requires science and not just intuition.
 
\subsection{slide-109}
Return to this amazing discovery.
 
Let me make some more points about it.
 
First, it doesn't fully answer our question about the relation between the Principles of Object
Perception and mechanisms in infants.  
It tells us that the Principles characterise a certain kind of perceptual process.
This is progress; but we can still ask about the nature of the procesess and representations 
involved.  This will become important when we consider knowledge in other domains.
 
Second, we haven't fully explained the discrepancy between looking and action-based measures for 
representing objects as persisting and tracking their causal interactions.
After all, why do these perceptual representations of objects--the object indexes--not guide 
purposive actions like reaching and pulling?
This is an issue we shall return to.
 
Third, it leaves us with a question we didn't have before.
What is the relation between these abilities to segment objects, represent them as persisting 
and track their causal interactions and knowledge about objects?
Clearly having an object-index stuck to an object is not the same thing as having knowledge
about the object's location and movements.  (If it were, we'd face just the problems that are 
fatal for the Simple View.)
What then is the relation between these things?
 
This third point is related to an issue about the relation between infant and adult capacities,
one that  I raised at the start of this lecture ...
 
\subsection{slide-110}
What is the relation between infants' competencies with objects and adults'?
Is it that infants' competencies grow into more sophisticated adult competencies?
Or is it that they remain constant throught development, and are supplemented by quite 
separate abilities?
 
\subsection{slide-114}
The identification of the Principles of Object Perception with object-indexes suggests that 
infants' abilities are constant throughout development.
They do not become adult conceptual abilities; rather they remain as perceptual systems
that somehow underlie later-developing abilities to acquire knowledge.
 
Confirmation for this view comes from considering that there are discrepancies in adults' 
performances which resemble the discrepancies in infants between looking and action-based 
measures of competence ...
[This links to unit 271 on perceptual expectations ...]
 
\subsection{unit\_271}
 
 
\section{Perceptual Expectations}
 
\subsection{slide-116}
Recall that the Principles of Object Perception generate expectations in infants.
For example, we saw (in the previous lecture) that infants expect objects to be in certain 
locations, or to appear at 
certain points in space.
 
What is the status of these expectations?
Our recent identification of the Principles of Object Perception with the operation of object
indexes suggests that the expectations are in some sense perceptual.
But what are perceptual expectations?
And do adults also have perceptual expectations?
 
\subsection{slide-117}
There are perceptual expectations.
 
Suppose you saw this image.
 
The triangle behind the thumb is in some sense perceptually present, even though you can't see it.
 
\subsection{slide-119}
But now the thumb comes away and what you see is not the triangle you were expecting.
 
Could these perceptual expectations be just a matter of knowledge?
 
No, because perceptual expectations are judgement-independent.
 
As \citep{kellman:1983_perception} report, Michotte, Thines and Crabbe found that subjects report seeing a single large triangle behind the thumb even when they know that there isn't one there.
 
You can cover and reveal the triangles repeatedly, but the expectation will hold firm.
 
 
 
The experience you have when the thumb is removed is like that of infants' in violation-of-expectation tasks.
 
This is what it is like to be an infant.
 
\subsection{slide-120}
You see an object move on a screen in some way.
 
The screen goes blank and you are asked to say where the object was last.
 
Subjects typically locate the object just a bit further on, as if it had continued moving after the screen went blank.
 
This is called representational momentum as a sort of joke; the idea is that the representation keeps moving if the object stop abruptly enough.
 
It's not interesting in itself, but it's useful for us.
 
Why is it useful?  Because it can tell us about the paths that the perceptual system expects objects to travel on.
 
\subsection{slide-121}
\citep{freyd:1994_representational}  ('The results are also consistent with a claim of relative cognitive impenetrability (Finke \& Freyd, 1989; Kelly \& Freyd, 1987) in that subjects showed a memory shift for a path that the majority of subjects did not consciously consider correct.' \citep[p.\ 975]{freyd:1994_representational})
 
'In Freyd and Jones’ study [(1994)], greater RM was observed for the impetus (spiral) than for the Newtonian (straight) path.'* (p. 449)
 
\subsection{slide-122}
The effect of mass on the rate of ascending motion: impetus and Newtonian theories come apart.
 
Important because it shows limits (people know better than their perceptual systems)
 
--- \citep{kozhevnikov:2001_impetus} on representational momentum
 
adults, including trained physicists who make correct verbal predictions about the effects of mass on motion, show representational momentum (a perceptual effect) consistent with impetus and inconsistent with Newtonian mechanics.  So again we have (i) judgement-independence and (ii) adults.  ('both physics experts and novices possess the same set of implicit beliefs about motion.' \citep[p.\ 451]{kozhevnikov:2001_impetus})
 
--- limits of infants' systems found in adults shows that we can identify the system as persisting
 
--- note that other studies also show judgement-independence, but in the other way (implicit knowledge of physical interactions more accurate than judgement: 'a number of previous studies suggesting a dissociation between explicit and implicit knowledge of the principles of physics. For instance, Hubbard suggests that implicit knowledge reflects internalization of invariant physical principles, whereas explicit knowledge about motion may be less accurate. Similarly, Krist et al. (1993) suggested that perceptually based knowledge is more accurate than verbal concepts of motion.' \citep[p.\ 450]{kozhevnikov:2001_impetus}
 
--- What questions does this bear on?  (1) perceptual \& modular nature of infants' understanding of objects and their interactions; (2) relation between infant competence and adults' (do the thing about system being transformed or discarded versus two systems persisting through development : this is different from issues about innateness, which looks from infants' competence backwards --- here we go in the other direction, looking forwards); and (3) trade-offs between flexibility and efficiency (see below; worth talking about this in some detail here)
 
--- on cognitive efficiency:  	'To extrapolate objects’ motion on the basis of physical principles, one should have assessed and evaluated the presence and magnitude of such imperceptible forces as friction and air resistance operating in the real world.  This would require a time-consuming analysis that is not always possible. In order to have a survival advantage, the process of extrapolation should be fast and effortless, without much conscious deliberation. Impetus theory allows us to extrapolate objects’ motion quickly and without large demands on attentional resources.' \citep[p.\ 450]{kozhevnikov:2001_impetus}
 
**caution (basically fine, but need to be careful): 'The extent to which displacement reflects physical principles per se has been widely debated in the literature; theories of displacement suggest a variety of potential effects of physical principles ranging from an incorporation of the principle of momentum into mental representation (e.g., Finke et al., 1986) to a rejection of any internalization of physical principles (e.g., Kerzel, 2000, 2003a). The empirical evidence is clear that (1) displacement does not always correspond to predictions based on physical principles and (2) variables unrelated to physical principles (e.g., the presence of landmarks, target identity, or expectations regarding a change in target direction) can influence displacement. ... information based on a naive understanding of physical principles or on subjective consequences of physical principles appears to be just one of many types of information that could potentially contribute to the displacement of any given target' \citep[p.\ 842]{hubbard:2005_representational}
 
\subsection{slide-123}
Why is it significant that representational momenutum in adults is consistent with
objects obeying impetus mechanics rather than Newtonian mechanics?
 
There are two reasons.
 
First, the signature limits provide us with a means to identify infant with adult cognition.
If both infants and adults are subjects to the same signature limits, namely those associated
with impetus mechanics, then we can infer that the representations of physical objects in 
adult 's perceptual processes originate from the same source as infants' representations of 
physical objects.
 
Second, it shows that the perceptual expectations involved in the representational momentum
are distinct from judgements.  You would not expect representational momentum to guide 
purposive actions that are separated in time from the perceptual experiences.
 
\subsection{slide-124}
Recall that in infants we found discrepancies between some looking and some action-based 
measures of abilities to track causal interactions.
While I don't think we are yet in a position to explain these discrepancies, we can identify 
them with discrepancies that are also found in adults.
I suggest that
the discrepancies bewteen perceptual expectations and verbal judgements in adults have the same
source
as the discrepancies bewteen some looking and action-based measures in infants.
Their explanation lies in the nature of perceptual expectations.
 
\subsection{slide-125}
If this is right, one consequence is that infants competences do not grow into more 
sophisticated adult competencies:  Rather they remain constant throught development, and 
are supplemented by quite separate abilities.
 
This, anyway, is the view I shall adopt from here on in.
 
\subsection{unit\_281}
 
 
\section{Knowledge of Objects: Conclusions}
 
\subsection{unit\_281-conclusions}
In this lecture I have been considering several interlocking themes simultaneously.
 
the Problem

          
knowledge of causal interactions

          
relation of infant to adult cognition
 
\subsection{slide-129}
Concerning knowledge of causal interactions, what have we learnt?
Infants, like adults, have perceptual systems that enable them to detect some kinds of 
causal interaction.
Now this does not mean, I think, that they have knowledge of particular causal interactions.
To attribute them such knowledge would conflict with their failures to act.
(Link this to the Hood et al experiment, intially presented as about permanence:
*TODO* **add three slides**: spelke ball drop, hood et al, leslie and keeble --- I want to
say these three all have a common explanation, namely perceptual mechanisms.)
 
\subsection{slide-130}
Recall this experiment about causal interactions.  Infants are more interested when the 
ball moves through the bench (the `inconsistent' condition).  Why?  Because the object
index stops at the solid barrier.
 
\subsection{slide-131}
Why do 2.5-year-olds look longer when experimenter removes the ball from behind the wrong 
door?  Because the object is attached to an object index behind another door.
But why don't they reach to the correct door?
Because having an object index attached to an object is not sufficient to intitate 
goal-directed action.
 
\subsection{slide-135}
Concerning knowledge of physical objects, there were several questions ...
First, Given that the simple view is wrong, what is the relation between the principles of object perception and infants’ competence in segmenting objects, object permanence and tracking causal interactions?
These principles describe the operation of a module whose function concerns objects.
The principles describe the operations of the system of object indexes.
(So the principles are not necessarily represented at all.)
 
Second, and relatedly, The principles of object perception result in ‘expectations’ in infants.  What is the nature of these expectations?
The expectations are perceptual expectations and the result of modular, computational processes.
The expectations are neither knowledge nor perceptions.
They are rather representations internal to the system of object indexes.
(This is why we get the two discrepancies between spontaneous looking and purposive actions 
like searching.)
 
\subsection{slide-138}
What is the relation between adults’ and infants’ abilities concerning physical objects and their causal interactions?
What is the relation between infants' and adults' capacities?  
Nonhuman adult primates plus representational momentum plus causal interaction : 
cautiously go for continuity (as in the case of categorical perception of colour)
 
\subsection{slide-139}
So earlier I contrasted these two views.
 
I think on balance the second is right.
 
Why?  *Nonhuman adult primates plus representational momentum plus causal interaction
 
\subsection{slide-144}
One last thing
There is a question how you get from core knowledge of objects to knowledge of objects.
We won't discuss this, but I think tool use is quite plausible.
Basic forms of tool use may not require understanding how objects interact 
(Barrett, Davis, \& Needham; Lockman, 2000), and may depend on core cognition of 
contact-mechanics (Goldenberg \& Hagmann, 1998; Johnson-Frey, 2004). Experience of 
tool use may in turn assist children in understanding notions of manipulation, a 
key causal notion (Menzies \& Price, 1993; Woodward, 2003). Perhaps non-core capacities 
for causal representation are not innate but originate with experiences of tool use.
 
\subsection{unit\_601\_milan}
 
 
\section{Core Knowledge}
 
\subsection{slide-146}
I talked about the notion of core knowledge in the very first lecture, but since then I 
have not appealed to the notion.
This is deliberate because the notion is tricky; so I thought it would be good to postpone
our discussion of it for as long as possible.
Now I can put it off no longer.
 
\subsection{slide-148}
The first, very minor thing is to realise that there are two closely related notions, core 
knowledge and core system.
These are related this: roughly, core knowledge states are the states of core systems.  More 
carefully:
 
For someone to have \textit{core knowledge of a particular principle or fact} is for her to have a core system where 
            either the core system includes a representation of that principle or else the principle plays a special role in describing the core system.
 
So we can define core knowlegde in terms of core system.
 
\subsection{slide-149}
What do people say core knowledge is?
 
There are two parts to a good definition.  The first is an analogy that helps us get a fix on what we is meant by 'system' generally.  (The second part tells us which systems are core systems by listing their characteristic features.)
 
\subsection{slide-151}
So talk of core knowledge is somehow supposed to latch onto the idea of a system.
 
What do these authors mean by talking about 'specialized perceptual systems'?
 
They talk about things like perceiving colour, depth or melodies.
 
Now, as we saw when talking about categorical perception of colour, we can think of the 'system' underlying categorical perception as largely separate from other cognitive systems--- we saw that they could be knocked out by verbal interference, for example.
 
So the idea is that core knowledge somehow involves a system that is separable from other cognitive mechanisms.
 
As Carey rather grandly puts it, understanding core knowledge will involve understanding something about 'the architecture of the mind'.
 
\subsection{slide-152}
Illustration: edge detection.
 
\subsection{slide-155}
 
This, them is the two part definition.  An analogy and a list of features.
 
\subsection{slide-156}
There is one more feature that I want to mention; this is important although I won't disucss it here.
 
To say that a represenation is iconic means, roughly, that parts of the representation represent parts of the thing represented.
 
Pictures are paradigm examples of representations with iconic formats.
 
For example, you might have a picture of a flower where some parts of the picture represent the petals and others the stem.
 
\subsection{slide-157}
The first problem we encountered was that the Simple View is false.
But maybe we can appeal to the Core Knowledge View.
 
According to the Core Knowledge View, the principles of object perception, and maybe also the 
expectations they give rise to, are not knowledge.
But they are core knowledge.
 
This raises some issues.  Is the Core Knowledge View consistent with the claims that 
we have ended up with, e.g. about categorical perception and the Principles of Object 
Perception characterising the way that object indexes work?
I think the answer is, basically, yes.  Categorical perception involves a system that has
many of the features associated with core knowledge.
 
But before going further, I think we critically examine the notion and see whether we really
understand what core knowledge is.
 
\subsection{slide-158}
As a sort of aside, I might mention for anyone familiar with Fodor on modularity 
that core systems basically coincide modules.
 
\subsection{slide-159}
Aside: compare the notion of a core system with the notion of a module
 
The two definitions are different, but the differences are subtle enough that we don't want both.
 
My recommendation: if you want a better definition of core system, adopt core system = module as a working assumption and then look to research on modularity because there's more of it.
 
An example contrasting Grice and Davidson on the wave.
 
\subsection{slide-167}
But is the notion of core system (or module) explanatory?
 
\subsection{slide-168}
One reason for doubting that the notion of a core system is explanator arises from the 
way we have introduced it.
We have introduced it by providing a list of features.
But why suppose that this particular list of features constitutes a natural kind?
This worry has been brought into sharp focus by criticisms of 'two systems' approaches.
(These criticisms are not directed specifically at claims about core knowledge, but the criticisms apply.)
 
‘there is a paucity of … data to suggest that they are the only or the best way of carving up the processing,
 
‘and it seems doubtful that the often long lists of correlated attributes should come as a package’
\citep[p.\ 759]{adolphs_conceptual_2010}
 
\subsection{slide-169}
‘we wonder whether the dichotomous characteristics used to define the two-system models are … perfectly correlated …
[and] whether a hybrid system that combines characteristics from both systems could not be … viable’
\citep[p.\ 537]{keren_two_2009}
 
\subsection{slide-171}
This is weak.
 
Remember that criticism is easy, especially if you don't have to prove someone is wrong.
 
Construction is hard, and worth more.
 
\subsection{slide-174}
Even so, there is a problem here.
 
‘the process architecture of social cognition is still very much in need of a detailed theory’
\citep[p.\ 759]{adolphs_conceptual_2010}
 
\subsection{slide-175}
Given the problems, maybe we should just abandon the notions of core system and core knowledge.
After all, we have come this far without them.
 
Why do we need a notion like core system?
 
\subsection{slide-176}
So why do we need a notion like core knowledge?
 
Think about these domains.
 
In each case, we're pushed towards postulating that infants know things, but also pushed against this.
 
Resolving the apparent contradiction is what core knowledge is for.
 
Key question: What features do we have to assign to core knowledge if it's to describe these discrepancies?
 
I think the fundamental feature is inaccessibility.
 
\subsection{slide-182}
If this is what core knowledge is for (if it exists to explain these discrepancies), what features must core knowledge have?
 
\subsection{slide-183}
Look at those features characterising core knowledge as defined by Carey and Spelke again
-- innate, encapsulated, unchanging and the rest.
None of these straightforwardly enable us to predict that core knowledge of objects will 
guide looking but not reaching.
(I'll say in a moment: 
The iconic format is important, but because it explains a missing feature, namely limited
accessibility.)
 
\subsection{slide-184}
The feature we most need is actually missing from their list.
 
To say that a system or module exhibits limited accessibility is to say that the representations in the system are not usually inferentially integrated with knowledge.
 
I think this is the key feature we need to assign to core knowledge in order to explain the apparent discrepancies in the findings about when knowledge emerges in development.
 
\subsection{slide-185}
Limited accessbility is a familar feature of many cognitive systems.
 
When you grasp an object with a precision grip, it turns out that there is a very reliable pattern.
 
At a certain point in moving towards it your fingers will reach a maximum grip aperture which is normally a certain amount wider than the object to be grasped, and then start to close.
 
Now there's no physiological reason why grasping should work like this, rather than grip hand closing only once you contact the object.
 
Maximum grip aperture shows anticipation of the object: the mechanism responsible for guiding your action does so by representing various things including some features of the object.
 
But we ordinarily have no idea about this.
 
The discovery of how grasping is controlled depended on high speed photography.
 
This is an illustration of limited accessibility.
 
(This can also illustrate information encapsulation and domain specificity.)
 
\subsection{slide-186}
To sum up, there are two problems with the notion of core knowledge ...
 
core systems -- questions:


          

            
Is appealing to core systems (modules) explanatory?

            
Can the Core Knowledge View explain the discrepancy?

            
Is the view I've been developing consisten with the Core Knowledge View?

            
Is core knowledge all one thing?

            
How do you get from core knowledge to knowledge knowledge?

          
 
The Core Knowledge View is the view that infants' competence with objects, causes, colours
and the rest depends not on knowledge (as the Simple View has it) but on core knowledge.
 
On the first question, I could tell you a long story about computational processes and 
representational formats.  But I won't do that here.
 
On the second question, I think the answer is probably yes because the difference in
representational formats between knowledge proper and core knowledge explains limited
accessibility, and this in turn partially explains the discrepancies.
(But note that it doesn't explain is why the discrepacies fall exactly where they do.)
 
On the third question, I think that thinking about categorical perception, or the perceptual 
systems that underpin objects indexes is a way of getting at what core knowledge really 
involves.  Core systems are not always something over and above perceptual and motor systems
(although it seems probably that they sometimes are if our competence with syntax rests on 
core knowledge).
 
On the fourth question,
Most people are committed to there being different stories for (a) colour (categorical 
perception); (b) syntax (tacit knowledge); and (c) physical objects.
I'm not sure. I'm tempted to think that perceptual and motor cognition are at the root of these 
in every case, but this is a radical idea for which there is currenly very little evidence.
 
The fifth question is just a variant of what I called The Problem (see lecture 2; it arises 
from the failure of the Simple View).
 
\subsection{slide-187}
This picture is significantly different from some competitors (but not Carey on number):
 
(1) because it shows we aren't done when we've explained the acquisition of core knowledge (contra e.g. Leslie, Baillargeon), and
 
(2) because it shows we can't hope to explain the acquisition of knowledge if we ignore core knowledge (contra e.g. Tomasello)
 

    
%--- end paste
%--------------- 
 





\bibliography{$HOME/endnote/phd_biblio}



\end{document}