 %!TEX TS-program = xelatex
%!TEX encoding = UTF-8 Unicode

%\def \papersize {a5paper}
\def \papersize {a4paper}
%\def \papersize {letterpaper}

%\documentclass[14pt,\papersize]{extarticle}
\documentclass[12pt,\papersize]{extarticle}
% extarticle is like article but can handle 8pt, 9pt, 10pt, 11pt, 12pt, 14pt, 17pt, and 20pt text

\def \ititle {Origins of Mind: Lecture Notes}
\def \isubtitle {Lecture 01}
%comment some of the following out depending on whether anonymous
\def \iauthor {Stephen A.\ Butterfill}
\def \iemail{s.butterfill@warwick.ac.uk% \& corrado.sinigaglia@unimi.it
}
%\def \iauthor {}
%\def \iemail{}
%\date{}

%\input{$HOME/Documents/submissions/preamble_steve_paper4}
\input{$HOME/Documents/submissions/preamble_steve_lecture_notes}

%no indent, space between paragraphs
\usepackage{parskip}

%comment these out if not anonymous:
%\author{}
%\date{}

%for e reader version: small margins
% (remove all for paper!)
%\geometry{headsep=2em} %keep running header away from text
%\geometry{footskip=1.5cm} %keep page numbers away from text
%\geometry{top=1cm} %increase to 3.5 if use header
%\geometry{bottom=2cm} %increase to 3.5 if use header
%\geometry{left=1cm} %increase to 3.5 if use header
%\geometry{right=1cm} %increase to 3.5 if use header

% disables chapter, section and subsection numbering
\setcounter{secnumdepth}{-1} 

%avoid overhang
\tolerance=5000

%\setromanfont[Mapping=tex-text]{Sabon LT Std} 


%for putting citations into main text (for reading):
% use bibentry command
% nb this doesn’t work with mynewapa style; use apalike for \bibliographystyle
% nb2: use \nobibliography to introduce the readings 
\usepackage{bibentry}

%screws up word count for some reason:
%\bibliographystyle{$HOME/Documents/submissions/mynewapa} 
\bibliographystyle{apalike} 


\begin{document}



\setlength\footnotesep{1em}






%--------------- 
%--- start paste
\title {Origins of Mind: Lecture Notes \\ Lecture 03}
 
\maketitle
 
 
\subsection{slide-3}
In this lecture I want to consider two interlocking themes simultaneously. The first is the problem I introduced in the last lecture, which arises from the failure of the Simple View. The problem is to understand the relation between infants' abilities regarding physical objects and the Spelke principles which describe these given that the principles and their deliverances are not actually knowledge. The other theme is knowledge of causal interactions. On causal interactions, my question is: How do infants first acquire knowledge of causal interactions among objects. I want to approach this question in an odd way, by thinking about the perception of causation in adults. My hope is that focussing on the perception of causation might help us to make progress with both themes simultaneously.
 
 
\subsection{slide-4}
I also want to complicate things by considering another question in this lecture. The question is, What is the relation between infants' competencies with objects and adults'? Is it that infants' competencies grow into more sophisticated adult competencies? Or is it that they remain constant throught development, and are supplemented by quite separate abilities?
 
 
\subsection{unit\_251}
 
\section{Perception of Causation}
 
 
\subsection{slide-10}
The question for this section is,
Can humans perceive causal interactions?
 
 
\subsection{slide-11}
adults: (a) verbal reports. So what?
 
 
\subsection{slide-12}
Adults will also report experiencing causal interactions including pullling, ...
 
 
\subsection{slide-14}
... disintegration ...
 
 
\subsection{slide-16}
... and bursting.
 
 
\subsection{slide-17}
But we shouldn't trust verbal reports. After all, people will say all kinds of things about their experiences. This is nicely illustrated by a famous experiment on apparent behaviour by \citet{Heider:1944ts}.
 
 
\subsection{slide-19}
Michotte: the experience of launching depends on interactions among various factors including

the relative speeds of the two objects
the delay between the first and second objects’ movements
the spatial gap between the two objects
the trajectories of the two objects.
But how does this help us? Importantly, tiny variations in the parameters will make big differences in the experiences reported. Let me illustrate this for the delay between the objects' movements.
 
 
\subsection{slide-20}
adults: (b) they can discriminate between short gaps and long gaps.
That is, the can discriminate gaps of around 50ms.
 
 
\subsection{slide-21}
Maybe this is clearer as a figure.
People can distinguish between stimuli that differ only in that the gap between two movements is approximately 50ms longer in one than the other. A 50ms difference makes the difference between reporting launching and reporting two movements.
We need to do more to understand the effect, ...
 
 
\subsection{slide-22}
... but first note that this effect, or one very like it, can also be found in infancy.
Of course, six-month-olds can't tell us about their expeirences. So how can we tell that they detect launching effects?
 
 
\subsection{slide-23}
Infants at around six months of age seem also to distinguish launching from other sequences, much as adults do \citep{Leslie:1987nr}.
[nb: Several people have discussed this in seminars so I won't discuss it here (the reference is on your handout).]
In their experiment, they compared two groups of infants. The first group was habituated to the top display, which is just the sort of animation Michotte used to get reports of causal experiences in adults. After habituation, this first group was then shown the same display except that the direction was reversed. Meanwhile a second group was habituated to a display like the top display here except that there was a delay between the first object stopping and the second object starting. This delay would mean that, in adults, there are no reports of experiences of a causal interaction. After habituation, this second group was shown the same display except that the direction of movement was reversed.
Of interest was whether the first group showed greater dishabituation to the reversal than the second group. How could this tell us anything about infants' experiences? Suppose that infants do not have anything like what adults report as an experience of causation. They they experience merely patterns of movement. And, in this case, reversing on sequence should create no more interest than reversing the other. But now suppose that infants do have something like what adults report as an experience of causation? Then, when reversing the first sequence, there are two changes: there is a change both to the movement and to the character of the causal interaction. To put it informally, reversing direction means that the patient of the interaction becomes its agent. So the hypothesis that infants' experiences of Michotte-like stimuli resemble adults predicts that there will be greater dishabituation when the first, `direct launching` sequence is reversed.
 
 
\subsection{slide-24}
And this is just what the researchers found. The table shows mean looking times in second (with standard deviations in brackets). The control group was just like the direct lanunching group except that there was no reversal
 
 
\subsection{slide-25}
The question was how we can get beyond intuition in understanding the verbal reports.
Part of the answer is this. We don't worry about the content of the verbal reports. We just focus on the fact that their content changes depending on a tiny, 50 millisecond difference in the delay between two movements. Call this \emph{launching effect}.
This doesn't tell us what people are detecting. But it does tell us that the effect is not merely confabulation or making it up. So we have taken a tiny step beyond intuition. But we also have to answer two questions.
 
 
\subsection{slide-26}
How is launching detected? For example, does it involve perceptual processes?
Why is a delay of up to around 70ms consistent with the launching effect occuring?
 
 
\subsection{slide-28}
We'll focus on the first question and come back to the second one later.
 
 
\subsection{slide-29}
So we have the launching effect: adults and probably infants too exhibit perceptual sensitivity to differences in timing of around 50 milliseconds, but only when such delays make the difference between a causal interaction or non-causal interaction.
We are still trying to understand the nature of the launching effect. To make progress we need to think about how it arises.
Guess how the launching effect works! A natural thought is this: first you perceive objects, then you identify causal interactions based on contiguity etc. This turns out to be completely wrong.
 
 
\subsection{slide-30}
The impression of launching is judgement-independent. So it can't be a consequence of thinking about the interaction. Still, it might be a consequence of perceiving objects in certain relations to each other. However a key finding shows that this is wrong. Surprisingly, we don't first perceive objects and then get the launching effect; rather, the launching effect is tied up with perceptual process of identifying objects' surfaces.
 
 
\subsection{slide-31}
illusory causal cresecents.
This depends on causal capture \citep{Scholl:2002eb}.
Normally, if the two balls overlap completely, subjects report seeing a single object changing colour.
 
 
\subsection{slide-33}
But if we show subjects a sequence like the launching effect but where the first square overlaps the second's position before it moves. When this event is shown is isolation almost all subjects see it as a single object changing colour. But when the event is shown with an unambiguous launching effect nearby, almost all subjects now see the 'overlap' event as a launching.
Causal capture means that we can show subjects a sequence with complete overlap and still have the report a causal effect.
‘when there is a launching event beneath the overlap (or underlap event) timed such that the launch occurs at the point of maximum overlap, observers inaccurately report that the overlap is incomplete, suggesting that they see an illusory crescent.’
\citep[p.\ 461]{Scholl:2004dx}
Why does the illusory causal crescent appear? Scholl and Nakayama suggest a
‘a simple categorical explanation for the Causal Crescents illusion: the visual system, when led by other means to perceive an event as a causal collision, effectively ‘refuses’ to see the two objects as fully overlapped, because of an internalized constraint to the effect that such a spatial arrangement is not physically possible. As a result, a thin crescent of one object remains uncovered by the other one-as would in fact be the case in a straight-on billiard-ball collision where the motion occurs at an angle close to the line of sight.’
\citep[p.\ 466]{Scholl:2004dx}
*here or later?
Contrast Spelke’s view.
‘objects are conceived: Humans come to know about an object’s unity, boundaries, and persistence in ways like those by which we come to know about its material composition or its market value.’
\citep[p.\ 198]{Spelke:1988xc}.
 
 
\subsection{slide-34}
(*This just shows when the overlap event was perceived as causal; not essential.)
 
 
\subsection{slide-35}
The question was,
How is launching detected? For example, does it involve perceptual processes?
Why is a delay of up to around 70ms consistent with the launching effect occuring?
[This question is blurred]
[This question is blurred]
We've just taken a step towards answering this first question, but more is needed.
 
 
\subsection{slide-37}
The second question remains open.
 
 
\subsection{slide-38}
Summary so far
The question for this section was:
Can humans perceive causal interactions?
Adults report experiencing a launching effect.
These reports show that they can make minute distinctions.
They can detect tiny, 50 millisecond differences in the delay between two movements.
Infants detect apparent causal interactions from 6 months of age or earlier.
Detecting apparent causal interactions is bound up with perceiving objects.
It's not that we first perceive objects and surfaces, and then identify causal interactions. Rather, identifying causal interactions is part of perceiving surfaces and objects. So the launching effect does seem to involve perceptual processes.
They can detect tiny, 50 millisecond differences in the delay between two movements.
It's not that we first perceive objects and surfaces, and then identify causal interactions. Rather, identifying causal interactions is part of perceiving surfaces and objects. So the launching effect does seem to involve perceptual processes.
This fourth point suggests we should step back and consider how objects are perceived. That is what we'll do next.
 
 
\subsection{unit\_261}
 
\section{Object Indexes and Causal Interactions}
 
 
\subsection{slide-41}
(from figure caption): ' A number (here eight) of identical objects are shown (at t = 1), and a subset (the `targets') is selected by, say, ̄ashing them (at t 􏰈 2), after which the objects move in unpredictable ways (with or without self-occlusion) for about 10 s. At the end of the trial the observer has to either pick out all the targets using a pointing device or judge whether one that is selected by the experimenter (e.g. by ̄ashing it, as shown at t 􏰈 4) is a target.' \citep[p.\ 142]{Pylyshyn:2001hl}
Highlight the case where subject is asked whether this is one of the objects identified.
(If a target disappears, subjects can also say where it was and which direction it was moving in.)
Limit of 3, maybe 4, objects will be important later.
 
 
\subsection{slide-47}
What does this tell us?
If attention is organised around objects, the perceptual system must be capable of identifying and tracking objects.
 
 
\subsection{slide-48}
Leslie et al say an object index is 'a mental token that functions as a pointer to an object' \citep[p.\ 11]{Leslie:1998zk}
'Pylyshyn's FINST model: you have four or five indexes which can be attached to objects; it's a bit like having your fingers on an object: you might not know anything about the object, but you can say where it is relative to the other objects you're fingering. (ms. 19-20)' \citep{Scholl:1999mi}
 
 
\subsection{slide-52}
Object indexes are linked to causation.
In order to track objects, a perceptual system has to be sensitive to be causal interactions
Why is this true?
Because when you have a causal interaction, there's a conflict between principles of object perception e.g. distinct surfaces=>two objects, vs good continuity of motion=>one object
The perceptual system needs to know when conflicts should be reconciled and when they should be written off.
We get perceptual effects of causal interactions when there are conflicts among cues of object identity.
This is a point Michotte made. He found that launching occurs when there is a conflict between cues to object identity: good continuity of movement suggests a single object whereas the existence of two distinct surfaces indicates two objects.
It is plausible that other types of causal interaction also involve conflicts between cues to object identity.
 
 
\subsection{slide-53}
Recall the question I asked earlier about 70ms.
Why is a delay of up to around 70ms consistent with the launching effect occuring?
This is an important question insofar as we are concerned with detecting causal interactions. Is what people detect when the launching effect occurs a causal interaction? You might say, it can't be because no delay between two movements is consistent with a causal interaction.
 
 
\subsection{slide-54}
Michotte said this:
‘anyone not very familiar with the procedure involved in framing the physical concepts of inertia, energy, conservation of energy, etc., might think that these concepts are simply derived from the data of immediate experience’ \citep[p.\ 223]{Michotte:1946nz}. 
How is this consistent with the laws of mechanics—surely no pause can be tolerated? Ingeniously, Michotte compares launching with the movement of a single object. The single object moves half way across a screen then pauses before continuing to move. Michotte found that the longest pause between the two movements consistent with subjects experiencing them as a single movement is around 80ms, exactly the longest pause consistent with experiences characteristic of launching \citep[pp.\ 91--8, 124]{Michotte:1946nz}. Accordingly, the experience characteristic of launching appears to require that the two movements be experienced as uninterrupted—this is why they can be separated by a pause of up to but no longer than 80ms.
 
 
\subsection{slide-55}
The question for this section was:
Can humans perceive causal interactions?
Now I think we have achieved an answer.
The perceptual system responsible for identifying objects must also concern itself with certain kinds of causal interaction in order to reconcile conflicting cues to object identity.
In slightly more detail: one function of our perceptual systems is to identify and track objects; this is done by means of various cues; sometimes the visual system is faced with conflicting cues to object identity which need to be resolved in order to arrive at a satisfactory interpretation; when certain types of causal interaction occur there is a conflict among cues to object identity; these conflicts must be treated differently from other conflicts because they do not indicate failures of object identification and so do not require resolution or further perceptual processing. So object perception depends on sensitivity to certain types of causal interaction and this is why the launching effect occurs.
 
 
\subsection{slide-56}
Before concluding I want to mention some further evidence for this view.
 
 
\subsection{slide-57}
This further evidence exploits something called the object-specific preview effect.
So before I can go on, I need to explain what this is.
 
 
\subsection{slide-58}
Background: object-specific preview effect
We can measure object indexes using the object-specific preview effect.
The \emph{object-specific preview effect}: ‘observers can identify target letters that matched the preview letter from the same object faster than they can identify target letters that matched the preview letter from the other object.’
\citep[p.\ 2]{Krushke:1996ge}
 
 
\subsection{slide-63}
Krushke and Fragassi (1996) have shown that the object-specific preview effect vanishes in launching but not in various spatio-temporally similar sequences. Since the object-specific preview effect is regarded as diagnostic of feature binding, this is evidence that in launching sequences, features of the second object (such as motion) remain bound to the first object for a short time after the second object starts to move.
 
 
\subsection{slide-64}
This is unexpected insofar as perception is often supposed to be limited to features of the world less abstract that causal interactions. Indeed, the notion that perceptual processes represent three-dimensional objects rather than mere surfaces was at one time controversial. The research we have reviewed shows that perceptual processes represent not only three-dimensional but properly physical objects, that is, objects capable of causally interacting with each other.
As we'll see in a moment, this is relevant to understanding a problem we encountered in the lecture on objects.
 
 
\subsection{unit\_266}
 
\section{Object Indexes and the Principles of Object Perception}
 
 
\subsection{slide-66}
This is a lecture about the origins of our knowledge of causal interactions, but I want to return to the topic of objects and the problem we encountered in the last lecture.
As I keep saying, knowledge of objects depends on abilities to (i) segment objects, (ii) represent them as persisting and (iii) track their interactions.
 
 
\subsection{slide-67}
When we asked how infants meet these three requirements, we found that a single set of principles, the Principles of Object Perception, seemed to underlie all three abilities.
We were then led to the question, What is the status of these principles? It's one thing to say that they describe how infants perform; but what we want is some understanding of the mechanisms.
 
 
\subsection{slide-68}
The Simple View is one way to get a mechanism out of the principles. Recall that the \emph{simple view} is the view that the principles of object perception are things that we know, and we generate expectations from these principles by a process of inference..
 
 
\subsection{slide-70}
Unfortunately, as we saw, the Simple View is wrong. We know it is wrong because it makes systematically incorrect predictions about infants' actions. These arise from a discrepancy between measures that involve looking times or eye movements and measures that involve other kinds of action, such as searching and pulling.
At the end of the last lecture (on objects), I said that the failure of the Simple View leaves us with a problem. We are now, at least, in a position to take a step towards solving that problem.
 
 
\subsection{slide-71}
The principles of object perception

are not items of knowledge

instead

they characterise the operation of

object-indexes (aka FINSTs, mid-level object files)

Their upshot is not knowledge about particular objects and their movements but rather a perceptual representation involving an object index.
\citep{Leslie:1998zk,Scholl:1999mi,Carey:2001ue}.
This amazing discovery is going to take us a while to fully digest. As a first step, note its significance for Davidson's challenge about characterising what is going on in the head of the child who has a few words, or even no words.
 
 
\subsection{slide-72}
We saw this quote in the first lecture ...
The discovery that the principles of object perception characterise the operation of object-indexes doesn't mean we have met the challenge exactly. We haven't found a way of describing the processes and representations that underpin infants' abilities to deal with objects and causes. However, we have reduced the problem of doing this to the problem of characterising how some perceptual mechanisms work. And this shows, importantly, that understanding infants' minds is not something different from understanding adults' minds, contrary to what Davidson assumes. The problem is not that their cognition is half-formed or in an intermediate state. The problem is just that understanding perception requires science and not just intuition.
 
 
\subsection{slide-73}
Return to this amazing discovery.
Let me make some more points about it.
First, it doesn't fully answer our question about the relation between the Principles of Object Perception and mechanisms in infants. It tells us that the Principles characterise a certain kind of perceptual process. This is progress; but we can still ask about the nature of the procesess and representations involved. This will become important when we consider knowledge in other domains.
Second, we haven't fully explained the discrepancy between looking and action-based measures for representing objects as persisting and tracking their causal interactions. After all, why do these perceptual representations of objects--the object indexes--not guide purposive actions like reaching and pulling? This is an issue we shall return to.
Third, it leaves us with a question we didn't have before. What is the relation between these abilities to segment objects, represent them as persisting and track their causal interactions and knowledge about objects? Clearly having an object-index stuck to an object is not the same thing as having knowledge about the object's location and movements. (If it were, we'd face just the problems that are fatal for the Simple View.) What then is the relation between these things?
This third point is related to an issue about the relation between infant and adult capacities, one that I raised at the start of this lecture ...
 
 
\subsection{slide-74}
What is the relation between infants' competencies with objects and adults'? Is it that infants' competencies grow into more sophisticated adult competencies? Or is it that they remain constant throught development, and are supplemented by quite separate abilities?
 
 
\subsection{slide-78}
The identification of the Principles of Object Perception with object-indexes suggests that infants' abilities are constant throughout development. They do not become adult conceptual abilities; rather they remain as perceptual systems that somehow underlie later-developing abilities to acquire knowledge.
Confirmation for this view comes from considering that there are discrepancies in adults' performances which resemble the discrepancies in infants between looking and action-based measures of competence ... [This links to unit 271 on perceptual expectations ...]
 
 
\subsection{unit\_271}
 
\section{Perceptual Expectations}
 
 
\subsection{slide-80}
Recall that the Principles of Object Perception generate expectations in infants. For example, we saw (in the previous lecture) that infants expect objects to be in certain locations, or to appear at certain points in space.
What is the status of these expectations? Our recent identification of the Principles of Object Perception with the operation of object indexes suggests that the expectations are in some sense perceptual. But what are perceptual expectations? And do adults also have perceptual expectations?
 
 
\subsection{slide-81}
There are perceptual expectations.
Suppose you saw this image.
The triangle behind the thumb is in some sense perceptually present, even though you can't see it.
 
 
\subsection{slide-83}
But now the thumb comes away and what you see is not the triangle you were expecting.
Could these perceptual expectations be just a matter of knowledge?
No, because perceptual expectations are judgement-independent.
As \citep{kellman:1983_perception} report, Michotte, Thines and Crabbe found that subjects report seeing a single large triangle behind the thumb even when they know that there isn't one there.
You can cover and reveal the triangles repeatedly, but the expectation will hold firm.
 
The experience you have when the thumb is removed is like that of infants' in violation-of-expectation tasks.
This is what it is like to be an infant.
 
 
\subsection{slide-84}
You see an object move on a screen in some way.
The screen goes blank and you are asked to say where the object was last.
Subjects typically locate the object just a bit further on, as if it had continued moving after the screen went blank.
This is called representational momentum as a sort of joke; the idea is that the representation keeps moving if the object stop abruptly enough.
It's not interesting in itself, but it's useful for us.
Why is it useful? Because it can tell us about the paths that the perceptual system expects objects to travel on.
 
 
\subsection{slide-85}
\citep{freyd:1994_representational} ('The results are also consistent with a claim of relative cognitive impenetrability (Finke \& Freyd, 1989; Kelly \& Freyd, 1987) in that subjects showed a memory shift for a path that the majority of subjects did not consciously consider correct.' \citep[p.\ 975]{freyd:1994_representational})
'In Freyd and Jones’ study [(1994)], greater RM was observed for the impetus (spiral) than for the Newtonian (straight) path.'* (p. 449)
 
 
\subsection{slide-86}
The effect of mass on the rate of ascending motion: impetus and Newtonian theories come apart.
Important because it shows limits (people know better than their perceptual systems)
--- \citep{kozhevnikov:2001_impetus} on representational momentum
adults, including trained physicists who make correct verbal predictions about the effects of mass on motion, show representational momentum (a perceptual effect) consistent with impetus and inconsistent with Newtonian mechanics. So again we have (i) judgement-independence and (ii) adults. ('both physics experts and novices possess the same set of implicit beliefs about motion.' \citep[p.\ 451]{kozhevnikov:2001_impetus})
--- limits of infants' systems found in adults shows that we can identify the system as persisting
--- note that other studies also show judgement-independence, but in the other way (implicit knowledge of physical interactions more accurate than judgement: 'a number of previous studies suggesting a dissociation between explicit and implicit knowledge of the principles of physics. For instance, Hubbard suggests that implicit knowledge reflects internalization of invariant physical principles, whereas explicit knowledge about motion may be less accurate. Similarly, Krist et al. (1993) suggested that perceptually based knowledge is more accurate than verbal concepts of motion.' \citep[p.\ 450]{kozhevnikov:2001_impetus}
--- What questions does this bear on? (1) perceptual \& modular nature of infants' understanding of objects and their interactions; (2) relation between infant competence and adults' (do the thing about system being transformed or discarded versus two systems persisting through development : this is different from issues about innateness, which looks from infants' competence backwards --- here we go in the other direction, looking forwards); and (3) trade-offs between flexibility and efficiency (see below; worth talking about this in some detail here)
--- on cognitive efficiency: 'To extrapolate objects’ motion on the basis of physical principles, one should have assessed and evaluated the presence and magnitude of such imperceptible forces as friction and air resistance operating in the real world. This would require a time-consuming analysis that is not always possible. In order to have a survival advantage, the process of extrapolation should be fast and effortless, without much conscious deliberation. Impetus theory allows us to extrapolate objects’ motion quickly and without large demands on attentional resources.' \citep[p.\ 450]{kozhevnikov:2001_impetus}
**caution (basically fine, but need to be careful): 'The extent to which displacement reflects physical principles per se has been widely debated in the literature; theories of displacement suggest a variety of potential effects of physical principles ranging from an incorporation of the principle of momentum into mental representation (e.g., Finke et al., 1986) to a rejection of any internalization of physical principles (e.g., Kerzel, 2000, 2003a). The empirical evidence is clear that (1) displacement does not always correspond to predictions based on physical principles and (2) variables unrelated to physical principles (e.g., the presence of landmarks, target identity, or expectations regarding a change in target direction) can influence displacement. ... information based on a naive understanding of physical principles or on subjective consequences of physical principles appears to be just one of many types of information that could potentially contribute to the displacement of any given target' \citep[p.\ 842]{hubbard:2005_representational}
 
 
\subsection{slide-87}
Why is it significant that representational momenutum in adults is consistent with objects obeying impetus mechanics rather than Newtonian mechanics?
There are two reasons.
First, the signature limits provide us with a means to identify infant with adult cognition. If both infants and adults are subjects to the same signature limits, namely those associated with impetus mechanics, then we can infer that the representations of physical objects in adult 's perceptual processes originate from the same source as infants' representations of physical objects.
Second, it shows that the perceptual expectations involved in the representational momentum are distinct from judgements. You would not expect representational momentum to guide purposive actions that are separated in time from the perceptual experiences.
 
 
\subsection{slide-88}
Recall that in infants we found discrepancies between some looking and some action-based measures of abilities to track causal interactions. While I don't think we are yet in a position to explain these discrepancies, we can identify them with discrepancies that are also found in adults. I suggest that the discrepancies bewteen perceptual expectations and verbal judgements in adults have the same source as the discrepancies bewteen some looking and action-based measures in infants. Their explanation lies in the nature of perceptual expectations.
 
 
\subsection{slide-89}
If this is right, one consequence is that infants competences do not grow into more sophisticated adult competencies: Rather they remain constant throught development, and are supplemented by quite separate abilities.
This, anyway, is the view I shall adopt from here on in.
 
 
\subsection{slide-91}
In this lecture I've been considering two interlocking themes simultaneously. The first is the problem I introduced in the last lecture, which arises from the failure of the Simple View.
 
 
\subsection{slide-93}
I suggested that a key step in solving this problem is the link between the Principles of Object Perception and object indexes. The Principles are not items of knowledge, nor do they generate knowledge about particular objects. Rather, the Principles characterise the operation of object indexes, and object indexes give rise to perceptual expectations. And, as we've just seen, perceptual expectations are not knowledge. The problem is to understand the relation between infants' abilities regarding physical objects and the Spelke principles which describe these given that the principles and their deliverances are not actually knowledge. The other theme is knowledge of causal interactions. On causal interactions, my question is: How do infants first acquire knowledge of causal interactions among objects. I want to approach this question in an odd way, by thinking about the perception of causation in adults. My hope is that focussing on the perception of causation might help us to make progress with both themes simultaneously.
 
 
\subsection{slide-96}
The second question was about the origins of our knowledge of causal interactions. How do humans first come to know about causes?
 
 
\subsection{slide-98}
I approached this question by asking a subquestion, Can humans perceive causal interactions? We've seen that, roughly speaking, the answer is yes. Humans, including infants from around 6 months or earlier, do have perceptual processes that function to track certain kinds of causal interactions among objects.
But how does this information about perceptual processes relate to knowledge? This is the Next Big Problem.
 
 
\subsection{slide-99}
Our Next Big Problem is this. We've said that infants' competence with causes and objects is not knowledge but something more primitive than knowledge, something which exists in adults too and can carry information discrepant with what they know. So, if at all, how does appealing to these early capacities enable us to explain the origins of knowledge?
 
 
\subsection{slide-100}
Broadly, my suggestion will be that the competence which appears in the first months of development leads to knowledge of objects and causes only in conjunction with various additional things, like social interaction, perhaps language and abilities to use tools.
The picture I want to offer differs from those of researchers like Vygotsky and Tomasello in that there is an essential role for early-developing forms of representation that are more primitive that concepts or thoughts and do not appear to have any kind of social origin.
But the picture also differs from those of researchers like Spelke and Carey in that these early developing forms of representation are only one of several components that are needed to understand the origins of knowledge.
To explore this idea I want to switch to a completely different domain, colour.
[*Aside on tool use:] Basic forms of tool use may not require understanding how objects interact (Barrett, Davis, \& Needham; Lockman, 2000), and may depend on core cognition of contact-mechanics (Goldenberg \& Hagmann, 1998; Johnson-Frey, 2004). Experience of tool use may in turn assist children in understanding notions of manipulation, a key causal notion (Menzies \& Price, 1993; Woodward, 2003). Perhaps non-core capacities for causal representation are not innate but originate with experiences of tool use.
%--- end paste
%--------------- 
 





\bibliography{$HOME/endnote/phd_biblio}



\end{document}