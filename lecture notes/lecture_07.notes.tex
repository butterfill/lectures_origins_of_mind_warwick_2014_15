 %!TEX TS-program = xelatex
%!TEX encoding = UTF-8 Unicode

%\def \papersize {a5paper}
\def \papersize {a4paper}
%\def \papersize {letterpaper}

%\documentclass[14pt,\papersize]{extarticle}
\documentclass[12pt,\papersize]{extarticle}
% extarticle is like article but can handle 8pt, 9pt, 10pt, 11pt, 12pt, 14pt, 17pt, and 20pt text

\def \ititle {Origins of Mind: Lecture Notes}
\def \isubtitle {Lecture 01}
%comment some of the following out depending on whether anonymous
\def \iauthor {Stephen A.\ Butterfill}
\def \iemail{s.butterfill@warwick.ac.uk% \& corrado.sinigaglia@unimi.it
}
%\def \iauthor {}
%\def \iemail{}
%\date{}

%\input{$HOME/Documents/submissions/preamble_steve_paper4}
\input{$HOME/Documents/submissions/preamble_steve_lecture_notes}

%no indent, space between paragraphs
\usepackage{parskip}

%comment these out if not anonymous:
%\author{}
%\date{}

%for e reader version: small margins
% (remove all for paper!)
%\geometry{headsep=2em} %keep running header away from text
%\geometry{footskip=1.5cm} %keep page numbers away from text
%\geometry{top=1cm} %increase to 3.5 if use header
%\geometry{bottom=2cm} %increase to 3.5 if use header
%\geometry{left=1cm} %increase to 3.5 if use header
%\geometry{right=1cm} %increase to 3.5 if use header

% disables chapter, section and subsection numbering
\setcounter{secnumdepth}{-1} 

%avoid overhang
\tolerance=5000

%\setromanfont[Mapping=tex-text]{Sabon LT Std} 


%for putting citations into main text (for reading):
% use bibentry command
% nb this doesn’t work with mynewapa style; use apalike for \bibliographystyle
% nb2: use \nobibliography to introduce the readings 
\usepackage{bibentry}

%screws up word count for some reason:
%\bibliographystyle{$HOME/Documents/submissions/mynewapa} 
\bibliographystyle{apalike} 


\begin{document}



\setlength\footnotesep{1em}






%--------------- 
%--- start paste
\title {Origins of Mind: Lecture Notes \\ Lecture 07}
 
\maketitle
 
 
\subsection{slide-3}
A key objection is this. In attempting to explain how infants first acquire knowledge of objects, causes and the rest, we want to invoke social interaction and, in particular, communication. But there's an objection to doing this. For, apparently communication depends on understanding minds and actions. And you might think that understanding minds and actions is something that comes late in development, well after there is knowledge of objects, causes and colours. In fact, you might even think that understanding minds and actions presupposes such things. If that were true, it would make no sense for us to appeal to communication, or social interaction, in explaining how humans first come to know things.
One aim of this and the next lecture is to reply to this objection. The reply is that there are forms of understanding minds and action which are more primitive than knowledge.
 
 
\subsection{slide-5}
Today's theme is understanding minds.
 
 
\subsection{unit\_401}
 
\section{Knowledge of Mind}
 
 
\subsection{slide-7}
The challenge is to explain the emergence, in evolution or development, of mindreading.
Let me explain.
 
 
\subsection{slide-10}
\textit{Mindreading} is the process of identifying mental states and purposive actions as the mental states and purposive actions of a particular subject.
 
 
\subsection{slide-11}
Researchers sometimes use the term ‘theory of mind’.
‘In saying that an individual has a theory of mind, we mean that the individual imputes mental states to himself and to others’
\citep[p.\ 515]{premack_does_1978}
 
 
\subsection{slide-14}
So, to be clear about the terminology, to have a theory of mind is just to be able to to mindread, that is, to identify mental states and purposive actions as the mental states and purposive actions of a particular subject.
 
 
\subsection{slide-18}
So the challenge is to explain the emergence of mindreading.
You know (let's say) that Ayesha belives Beatrice is in the library.
Humans are not born knowing individuating facts about others' beliefs.
How do they come to be in a position to know such facts?
Meeting this challenge initially seems simple.
But, as you'll see, we quickly end up with a puzzle.
I think this puzzle requires us to rethink what is involved in having a conception of the mental.
 
 
\subsection{slide-19}
I shall focus on awareness of others' beliefs to the exclusion of other mental states.
There's no theoretical reason for this; it's just a practical thing.
And what we learn about belief will generalise to other mental states.
 
 
\subsection{maxi\_story}
How can we test whether someone is able to ascribe beliefs to others?
Here is one quite famous way to test this, perhaps some of you are even aware of it already.
Let's suppose I am the experimenter and you are the subjects.
First I tell you a story ...
 
 
\subsection{slide-24}
Here's the really surprising thing.
Children do really badly on this until they are around four years of age.
And they seem to develop the ability to pass this task only gradually, over months or years.
(There's something else that isn't surprising to most people but should be: adult humans not only nearly always provide the answer we're calling 'correct': they also believe that there is an obviously correct answer and that it would be a mistake to give any other answer. I'll return to this point later.)
(NB: The figure is not Wimmer \& Perner's but drawn from their data.)
 
 
\subsection{slide-25}
There's been some stuff in the press recently about bad science, mainly some dodgy methods and failures to replicate.
 
 
\subsection{slide-26}
So you'll be pleased to know that a meta-study of 178 papers confirmed Wimmer \& Perner's findings.
Now there is clearly some variation here.
That's because different researchers implemented different versions of the original task.
We can use the meta-analysis of these experiments as a shortcut to finding out what sorts of factors affect children's performance.
 
 
\subsection{slide-27}
One factor that seems to make hardly any difference is whether you ask children about others' beliefs or their own beliefs.
To repeat, you get essentially the same results whether you ask children about others' beliefs or their own beliefs.
Children literally do not know their own minds.
 
 
\subsection{slide-28}
What happens if we involve the child by having her interact with the protagonist?
The task becomes easier for children of all ages, but the transition is essentially the same (participation does not interact with age \citealp[pp.\ 665-7]{Wellman:2001lz}).
 
 
\subsection{slide-29}
Finally, although there are some cultural differences, you get the same transition in seven diferent countries.
 
 
\subsection{slide-30}
So our challenge was to explain the emergence of mindreading.
At this point, up until around, it seemed quite straightforward to most researchers.
We seemed to know that children are unaware of mental states until around four years.
And a lot of studies looked at which factors affect their acquiring this awareness.
These studies showed that executive function, language and rich forms of social interaction are all important.
All of this supported something like the story that Sellars tells in his famous Myth of Jones.
 
 
\subsection{slide-31}
*todo*: describe Sellars' myth; link to Gopnik theory theory idea.
 
 
\subsection{slide-32}
But there was a big surprise in store for us.
 
 
\subsection{unit\_411}
 
\section{Mindreading: First Puzzle}
 
 
\subsection{slide-34}
There is a puzzle about when humans can first know individuating facts about others' beliefs.
To understand the origins of this knowledge we need to understand the puzzle.
So I'm going to reveal the puzzle to you. But let me start with a bit of background.
 
 
\subsection{slide-35}
Recall the experiment that got us started.
These experimenters added an anticipation prompt and measured to which box subjects looked first \citep{Clements:1994cw}.
(Actually they didn't use this story; theirs was about a mouse called Sam and some cheese, but the differences needn't concern us.)
 
 
\subsection{slide-40}
What got me hooked philosophical psychology, and on philosophical issues in the development of mindreading in particular was a brilliant finding by Wendy Clements who was Josef Perner's phd student.
These findings were carefully confirmed \citep{Clements:2000nc,Garnham:2001ql,Ruffman:2001ng}.
Around 2000 there were a variety of findings pointing in the direction of a confict between different measures.
These included studies on word learning \citep{Carpenter:2002gc,Happe:2002sr} and false denials \citep{Polak:1999xr}.
But relatively few people were interested until ...
 
 
\subsection{slide-92}
The challenge is to explain the emergence, in evolution or development, of mindreading.
Initially it looked like this was going to be relatively straightforward and involve just language, social interaction and executive function.
So a Myth of Jones style story seemed viable.
But the findings of competence in infants of around one year of age changes this.
These findings tell us that not all abilities to represent others' mental states can depend on things like language.
And, as I've been stressing, these findings also create a puzzle.
The puzzle is, roughly, how to reconcile infants' competence with three-year-olds' failure.
 
 
\subsection{slide-93}
The puzzle is a little bit like the puzzle we had in the case of knowledge of physical objects.
But it's also different.
In the case of physical objects, the conflict was between measures involving looking and measures involving searching.
In this case it's different, because on the infant side there is not just looking but also acting (e.g. helping) and even communicating.
 
*todo*: There are at least two possible puzzles you might focus on:
1. How can we avoid the apparent contradiction in the evidence?
2. How can we explain the apparent discrepancy between infants and 3-year-olds' performances?
Minimal theory of mind might resolve puzzle 1 but it won't by itself resolve puzzle 2.
This wasn't clear enough in the lecture (sorry!).
 
 
\subsection{slide-94}
Can we solve the puzzle by appeal to core knowledge (or 'modularity')?
The difference in measures is a hopeful sign that we can.
But the fact that representations of others' minds influence 1-year-olds' actions (e.g. in communicating and helping) complicates things because we imagine modules as inferentially isolated from practical reasoning.
Looking at a further puzzle will help us.
 
 
\subsection{unit\_421}
 
\section{Mindreading: Second Puzzle}
 
 
\subsection{slide-113}
So how does this second puzzle bear on our overall objective?
The challenge is to explain the emergence, in evolution or development, of mindreading.
The fact that mindreading is sometimes automatic and sometimes not is good evidence that mindreading is not a single thing.
Rather there are multiple kinds of process or system involved in mindreading.
From this we can conclude two things.
First, in this respect, the cases of mindreading is much like the cases of colour and physical objects.
These also involved multiple systems; this took the form of a distinction between core knowledge and knowledge proper.
Second, we have further grounds for thinking that some forms of mindreading are modular.
 
 
\subsection{slide-114}
[These slides are from a joint presentation with Ian Apperly (BCCCD)]
What is the relation between infant and adult capacities for mindreading?
It's important at this point to recognise that there are two quite different possibilties.
 
 
\subsection{unit\_431}
 
\section{Modules and Cognitive Efficiency}
 
 
\subsection{slide-121}
We can resolve both puzzles by appeal to the idea that there are modular \& non-modular mindreading processes. (Cf. physical objects).
 
 
\subsection{slide-122}
Explain how this works in the case of each puzzle.
*todo* modify the slide to illustrate the solution
 
 
\subsection{slide-126}
First, what do I mean by cognitive efficiency.
Some tasks require mental effort.
For example, suppose I ask you to count from one to one hundred omitting each prime number.
This is not the kind of thing most people can do while washing up.
It requires attention, inhibitory control and working memory.
By contrast, many adults can count in the ordinary way from one to one hundred while washing up.
Counting has become routine, habitual.
 
 
\subsection{slide-129}
So much for cognitive efficiency.
Now why suppose that modularity (or core knowledge) requires cognitive efficiency?
To invoke modularity, we need to understand how mindreading could be (i) automatic and (ii) present in pre-linguistic infants with limited working memory \& executive function; both (i) and (ii) mean we need to understand how it could be cognitively efficient. (The situation is a bit like this: we want to say you can perceive others mental states; but on the face of it, mental states are exactly the sort of things that are not available to perception.)
 
 
\subsection{slide-130}
We saw earlier that mindreading in four year olds and adults is cognitively demanding.
And there's good reason to think that it should be.
If anything should consume scarce cognitive resources ...
Now appeal to modularity doesn't explain how mindreading might somehow be efficient.
Suppose someone could find prime factors incredibly quickly.
It wouldn't be a satisfying explanation to just say that she had a module for finding prime factors.
We'd also need an algorithm that her module could be implementing consistently with her performance.
So (a) efficiency points to modularity but (b) efficiency requires explanation and (c) gesturing at modularity doesn't explain efficiency.
To see how mindreading could be cognitively efficient, we need to reject a dogma.
 
 
\subsection{unit\_441}
 
\section{Minimal Theory of Mind}
 
 
\subsection{unit\_451}
 
\section{Signature Limits Generate Predictions}
 
 
\subsection{slide-173}
So let me conclude.
The challenge we have been addressing was to understand the emergence of mindreading.
Initially this seemed straightforward: you learn this from social interaction using language as a tool (compare Gopnik's theory theory).
However, the discovery that abilities to track beilefs exist in infants from around 7 months or earlier initially suggested a different picture:
one on which mindreading was likely to involve core knowledge.
But, as always, things are not so straightforward. The evidence is apparently conflicting.
There are actually two conflicts, not one: developmentally (A- \& B-tasks) and in adults, automaticity.
The existence of two puzzles gives us confidence that the conflict is not merely a methodological artefact.
The solution is to recognise that there are modules, but there was an obstacle to the hypothesis that mindreading could be modular (*cognitive efficiency)
In constructing minimal theory of mind we've earned the right to solve them by appeal to modularity. (NB: it's modularity rather than mTm that explains the discrepancy; mTm is important because (i) it explains efficiency; and (ii) it generates predictions via signature limits)
Now the idea that there are both modular and non-modular mindreading enables us to solve the two puzzles (developmentally (A- \& B-tasks) and in adults, automaticity.).
However, this resolution of the puzzles doesn't answer our overall challenge about the origins of knowledge of other minds.
In fact it complicates the account of the origins of knowledge of other minds, makes the challenge seem harder rather than easier to meet.
We can't give a theory theory / learning account; and we also can't give a straightforward core knowledge account.
Instead we need something a bit more complex.
I haven't tried to offer an account of what that thing is yet, and that isn't the point of this lecture.
But let me close by describing how I would approach it.
A key issue is the relation between infant and adult competence ...
 
 
\subsection{slide-174}
Infant competence retained in adults (matching signature limits from Low \& Watts), as in the cases of colour (and probably physical objects too).
Now if we accept this, it is tempting to conjecture that the later developing mindreading abilities involve a process of rediscovery ...
 
 
\subsection{slide-175}
- theme A: explain the origins of knowledge of others minds : development as rediscovery. There is a modular capacity ( = core knowledge).
 
 
\subsection{slide-176}
development as rediscovery.
Let me pause to evalaute the picture I offered earlier in the light of what we've learnt so far.
(I hesitate to do this because it shows the picture I offered you isn't very good.)
Take each case in turn.
For colour it works quite well, providing, as I suggested last week, that acquiring words is a creative process of social interaction.
What about physical objects? Here there's no indication that using labels for objects drives a developmental change, and it's hard to see why it would.
(It's more plausible that tool use rather than word use matters; but even this is is hugely speculative.)
So no marks for that case at all.
What about minds, and in particular beliefs?
Superficially things look better here. There is both evidence that rich forms of social interaction facilitate development \citep{Hughes:2006fu}
and also evidence that language matters in various ways \citep{Astington2005ot}.
But these probably don't connect in the simple way I envisage.
Social interaction might matter because it provides experiences of perspective differences, or because it motivates children to think about others' minds.
And language might matter because having sentences around enables them to keep track of beliefs, or because using relative clauses might clue them in to a relation between beliefs and what utterances of sentences express.
So here the picture isn't right, but it might not be a million miles off either ...
It's wrong to think that labelling beliefs matters; but it may be that being able to talk about beliefs (implicitly or otherwise) does matter for coming to have knowledge of them.

%--- end paste
%--------------- 
 





\bibliography{$HOME/endnote/phd_biblio}



\end{document}